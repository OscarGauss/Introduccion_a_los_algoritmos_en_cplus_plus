%----------------------------------------------------------------------------------------
%	CHAPTER 3
%----------------------------------------------------------------------------------------

\chapter{Programación modular}
\renewcommand{\chaptername}{Lecture}
\section{Introduccion}\index{Introduccion}

\subsection{¿Qué es la Programación Modular?}\index{Introduccion!¿Qué es la Programación Modular?}


La programacion modular es un paradigma (modelo) de programacion, que consiste en dividir un problema en subproblemas con el fin de simplificarlo.

Aplicando la Programacion Modular podemos llevar problemas grandes y tediosos, a pequeños subproblemas, y estos a su ves en otros, hasta poder resolverlos facilmente con un lenguaje de programacion, a esta tecnica la llamaremos divide y venceras.

Un modulo es cada una de las partes del programa que resuelve un subproblema en las que se dividio el problema original.

\section{Modulos}\index{Modulos}

\subsection{Concepto de Modulo}\index{Modulos!Concepto de Modulo}

Un Modulo es una segmento de codigo separado del bloque principal que puede ser invocado en cualquier momento desde este o desde otro modulo.

\subsection{Elementos de declaracion del Modulo}\index{Modulos!Elementos de declaracion del Modulo}

Un Moduo o subrutina se declara, generalmente, por:
\begin{itemize}
\item Un nombre unico con el que se lo identifica y distingue de otros.
\item Un tipo de dato de retorno.
\item Una lista de parametros, puede ser cero, uno o varios.
\item Conjunto de ordenes que debe ejecutar la subrutina.
\end{itemize}

\subsection{Ejemplos}\index{Modulos!Ejemplos}

El siguiente ejemplo muestra una subrutina que tiene un parametro de entrada (x) del cual calcula su factorial y lo devuelve como parametro de salida.

El factorial de un numero n se define como el producto de todos los numeros desde 1 hasta n, y se simboliza n!. 
\[
5!=1x2x3x4x5=120
\]

\begin{lstlisting}[style=Cpp, label=fact, caption=Factorial]
#include <iostream>  
using namespace std;
//"int" Tipo de dato de retorno.
//"fact" Nombre de la subrutina.
//"(int x)" Paranetros que recibe.
int fact(int x){
	//Ordenes que ejecuta
	int f=1
	for(int i=1; i<=x;i++)
		f=f*i;
	return f;
}
int main(){
	int n;
	cin>>n;
	cout<<fact(n)<<endl;
	return 0;
}
\end{lstlisting}

\section{Recursividad}\index{Recursividad}

\subsection{Que es recursvidad?}\index{Recursividad!Que es recursividad?}

La recursividad es una tecnica de programacion en la que un Modulo hace una llamada a si mismo con el fin de resolver el problema. La llamada a si mismo se conoce como llamada recursiva.

Dicho formalmente, un algoritmo se dice recursivo si calcula instancias de un problema en función de otras instancias del mismo problema hasta llegar a un caso base, que suele ser una instancia pequeña del problema, cuya respuesta generalmente está dada en el algoritmo y no es necesario calcularla.

Aun asi la definicion puede ser confusa, por eso ahora presentaremos 2 ejemplos que ilustran la recursividad:

\subsection{Ejemplos}\index{Recursividad!Ejemplos}
\begin{itemize}
\item Al igual que en el primer ejemplo tendremos una subrutina o funcion que calcule el factorial de un numero, en este caso lo hara de manera recursiva.

\begin{lstlisting}[style=Cpp, label=factrec, caption=Factorial Recursivo]
#include <iostream>  
using namespace std;

int fact(int n){
	/*
	Caso base: donde el algoritmo se detendra por que tendra
	un valor conocido y no necesitara calcularlo pues 1!=1
	*/
	if(n==1) return 1;
	/*
	Si el valor es diferente de 1, entonces es un valor que 
	necesitamos calcular, por la definicion de factorial
	podemos apreciar que, por ejemplo: 5!=5x4! y que a su 
	vez 4!=4*3! y asi sucesivamente, esto es recursividad 
	ya que esta funcion se llama asi misma para calcular el
	siguiente factorial.
	*/
	return n*fact(n-1);
}
int main(){
	int n;
	cin>>n;
	cout<<fact(n)<<endl;
	return 0;
}
\end{lstlisting}

\item Ahora calcularemos el n-simo termino de la sucesion Fibonacci por medio de un algoritmo recursivo.

La sucecion Fibonacci, comienza con los valores 1 y 1 y a partir de estos cada termino es la suma de los 2 anteriores:
\[
1,1,2,3,5,8,13,21,34,...
\]

\begin{lstlisting}[style=Cpp, label=fibo, caption=Sucesion Fibonacci]
#include <iostream>  
using namespace std;

int Fibo(int n){
	//Caso base: El primer y segundo termino de la sucesion
	//son 1.
	if(n==1 or n==2) return 1;
	//Por la definicion de la sucecion un termino es igual a 
	//la suma de los 2 terminos anteriores.
	return Fibo(n-1)+Fibo(n-2);
}
int main(){
	int n;
	cin>>n;
	cout<<Fibo(n)<<endl;
	return 0;
}
\end{lstlisting}
\end{itemize}


