%----------------------------------------------------------------------------------------
%	CHAPTER 1
%----------------------------------------------------------------------------------------

%\chapterimage{chapter_head_1.pdf} % Chapter heading image

\chapter{Introducción a la programación en c++}

\section{¿Qué es un Lenguaje de Programación?}\index{¿Qué es un Lenguaje de Programación?}

Antes de hablar de C++, es necesario explicar que un lenguaje de programación es una herramienta que nos permite comunicarnos e instruir a la computadora para que realice una tarea específica. Cada lenguaje de programación posee una sintaxis y un léxico particular, es decir, forma de escribirse que es diferente en cada uno por la forma que fue creado y por la forma que trabaja su compilador para revisar, acomodar y reservar el mismo programa en memoria.

Existen muchos lenguajes de programación de entre los que se destacan los siguientes:

\begin{itemize}
\item C
\item C++
\item Basic
\item Ada
\item Java
\item Pascal
\item Python
\item Fortran
\item Smalltalk
\end{itemize}


\section{Historia de C++}\index{Historia de C++}

C++ es un lenguaje de programación creado por Bjarne Stroustrup en los laboratorios de At&T en 1983. Stroustrup tomó como base el lenguaje de programación más popular en aquella época el cual era C.

El C++ es un derivado del mítico lenguaje C, el cual fue creado en la década de los 70 por la mano del finado Dennis Ritchie para la programación del sistema operativo (un sistema parecido a Unix es GNU/Linux), el cual surgió como un lenguaje orientado a la programación de sistemas (System Programming) y de herramientas (Utilities) recomendado sobre todo para programadores expertos, y que no llevaba implementadas muchas funciones que hacen a un lenguaje más comprensible.

Sin embargo, aunque esto en un inicio se puede convertir en un problema, en la práctica es su mayor virtud, ya que permite al programador un mayor control sobre lo que está haciendo. Años más tarde, un programador llamado Bjarne Stroustrup, creo lo que se conoce como C++.

Necesitaba ciertas facilidades de programación, incluidas en otros lenguajes pero que C no soportaba, al menos directamente, como son las llamadas clases y objetos, principios usados en la programación actual. Para ello rediseñó C, ampliando sus posibilidades pero manteniendo su mayor cualidad, la de permitir al programador en todo momento tener controlado lo que está haciendo, consiguiendo así una mayor rapidez que no se conseguiría en otros lenguajes.

C++ pretende llevar a C a un nuevo paradigma de clases y objetos con los que se realiza una comprensión más humana basándose en la construcción de objetos, con características propias solo de ellos, agrupados en clases. Es decir, si yo quisiera hacer un programa sobre animales, crearía una clase llamada animales, en la cual cada animal, por ejemplo un pato, sería un objeto, de tal manera que se ve el intento de esta forma de programar por ser un fiel reflejo de cómo los humanos (en teoría) manejamos la realidad.

Se dice que nuestro cerebro trabaja de forma relacional (relacionando hechos), es por ello que cada vez que recuerdas algo, (cuentas un hecho), termina siendo diferente (se agregan u omiten partes).


\section{Qué es C++}\index{Qué es C++}

C++ es un lenguaje de programación orientado a objetos que toma la base del lenguaje C y le agrega la capacidad de abstraer tipos como en Smalltalk.

La intención de su creación fue el extender al exitoso lenguaje de programación C con mecanismos que permitieran la manipulación de objetos. En ese sentido, desde el punto de vista de los lenguajes orientados a objetos, el C++ es un lenguaje híbrido.

Posteriormente se añadieron facilidades de programación genérica, que se sumó a los otros dos paradigmas que ya estaban admitidos (programación estructurada y la programación orientada a objetos). Por esto se suele decir que el C++ es un lenguaje de programación multiparadigma.


\section{Herramientas Necesarias}\index{Herramientas Necesarias}

Las principales herramientas necesarias para escribir un programa en C++ son las siguientes:

\begin{enumerate}
\item Un equipo ejecutando un sistema operativo.
\item Un compilador de C++
	\begin{itemize}
	\item Windows MingW (GCC para Windows) o MSVC (compilador de microsoft con versión gratuita)
	\item Linux (u otros UNIX): g++
	\item Mac (con el compilador Xcode)
	\end{itemize}
\item Un editor cualquiera de texto, o mejor un entorno de desarrollo (IDE)
	\begin{itemize}
	\item Windows:
		\begin{itemize}
		\item Microsoft Visual C++ (conocido por sus siglas MSVC). Incluye compilador y posee una versión gratuita (versión express)
		\item Bloc de notas (no recomendado)
		\item Editor Notepad++
		\item DevCpp (incluye MingW - en desuso, no recomendado, incluye también un compilador)
		\item Code::Blocks
		\end{itemize}
	\item Linux (o re-compilación en UNIX):
		\begin{itemize}
		\item Gedit
		\item Kate
		\item KDevelop
		\item Code::Blocks
		\item SciTE
		\item GVim
		\end{itemize}
	\item Mac:
		\begin{itemize}
		\item Xcode (con el compilador trae una IDE para poder programar)
		\end{itemize}
\item Tiempo para practicar
\item Paciencia
\end{enumerate}
\section{Citation}\index{Citation}

This statement requires citation \cite{book_key}; this one is more specific \cite[122]{article_key}.

%------------------------------------------------

\section{Lists}\index{Lists}

Lists are useful to present information in a concise and/or ordered way\footnote{Footnote example...}.

\subsection{Numbered List}\index{Lists!Numbered List}

\begin{enumerate}
\item The first item
\item The second item
\item The third item
\end{enumerate}

\subsection{Bullet Points}\index{Lists!Bullet Points}

\begin{itemize}
\item The first item
\item The second item
\item The third item
\end{itemize}

\subsection{Descriptions and Definitions}\index{Lists!Descriptions and Definitions}

\begin{description}
\item[Name] Description
\item[Word] Definition
\item[Comment] Elaboration
\end{description}
