%----------------------------------------------------------------------------------------
%	PACKAGES AND OTHER DOCUMENT CONFIGURATIONS
%----------------------------------------------------------------------------------------

\documentclass[11pt,fleqn]{book} % Default font size and left-justified equations

\usepackage[top=3cm,bottom=3cm,left=3.2cm,right=3.2cm,headsep=10pt,paperwidth=8.5in,paperheight=11in]{geometry} % Page margins

\usepackage{xcolor} % Required for specifying colors by name

\definecolor{mycolor}{RGB}{24,97,243} % Define the orange color used for highlighting throughout the book
\usepackage{color}
\definecolor{mygreen}{rgb}{0,0.6,0}
\definecolor{mygray}{rgb}{0.5,0.5,0.5}
\definecolor{mymauve}{rgb}{0.58,0,0.82}
\definecolor{greencodeblocks}{RGB}{0,160,0}
\definecolor{graycodeblocks}{RGB}{160,160,160}
\definecolor{bluecodeblocks}{RGB}{0,0,255}

% Font Settings
\usepackage{avant} % Use the Avantgarde font for headings
%\usepackage{times} % Use the Times font for headings
\usepackage{mathptmx} % Use the Adobe Times Roman as the default text font together with math symbols from the Sym­bol, Chancery and Com­puter Modern fonts

\usepackage{microtype} % Slightly tweak font spacing for aesthetics
\usepackage[utf8]{inputenc} % Required for including letters with accents
\usepackage[T1]{fontenc} % Use 8-bit encoding that has 256 glyphs

\usepackage{listings} % Para estrofas de codigo
\usepackage{caption} % Para captar label y caption del lstlisting o listing

%%%%%%%%%%%%%%%%%%%%%%%
\definecolor{gray97}{gray}{.97}
\definecolor{gray75}{gray}{.75}
\definecolor{gray45}{gray}{.45}
 
\usepackage{listings}
\lstset{ frame=Ltb,
     framerule=0pt,
     aboveskip=0.5cm,
     framextopmargin=3pt,
     framexbottommargin=3pt,
     framexleftmargin=0.4cm,
     framesep=0pt,
     rulesep=.4pt,
     backgroundcolor=\color{mycolor!5},
     rulesepcolor=\color{black},
     %
     stringstyle=\ttfamily,
     showstringspaces = false,
     basicstyle=\small\ttfamily,
     commentstyle=\color{graycodeblocks},
     keywordstyle=\bfseries\color{greencodeblocks},
     %identifierstyle=\color{blue},
     stringstyle=\color{bluecodeblocks},
     %
     numbers=left,
     numbersep=15pt,
     numberstyle=\small\color{mycolor},
     numberfirstline = false,
     breaklines=true,
     %
     texcl=true, %para abilitar las tildes en el comentario
   }

% minimizar fragmentado de listados
\lstnewenvironment{listing}[1][]
   {\lstset{#1}\pagebreak[0]}{\pagebreak[0]}
 
\lstdefinestyle{consola}
   {basicstyle=\scriptsize\bf\ttfamily,
    backgroundcolor=\color{gray75},
   }
 
\lstdefinestyle{Cpp}
   {language=C++,
   }

%%%%%%%%%%%%%Aumentar sub sub section 

%%%%%%%%%%%%%%%%%
\newcommand{\tab}[1]{\hspace{.2\textwidth}\rlap{#1}} % nuevo comando para tab :)
%%%%%%%%%%%%%%%%%

%%%%%%%%%%%%%%%%%%%%%%%%%%%%5

% Bibliography
\usepackage[style=alphabetic,sorting=nyt,sortcites=true,autopunct=true,babel=hyphen,hyperref=true,abbreviate=false,backref=true,backend=biber]{biblatex}
\addbibresource{bibliography.bib} % BibTeX bibliography file
\defbibheading{bibempty}{}

% Index
\usepackage{calc} % For simpler calculation - used for spacing the index letter headings correctly
\usepackage{makeidx} % Required to make an index
\makeindex % Tells LaTeX to create the files required for indexing

%----------------------------------------------------------------------------------------

\input{structure} % Insert the commands.tex file which contains the majority of the structure behind the template


\begin{document}

%Cambiando al español :p o a lo que queres

%\renewcommand*\contentsname{Contenidos} %Para cambiar el nombre por defecto'Contents' por Contenidos aqui por alguna razon no cambia esta en content
\renewcommand*{\lstlistingname}{Codigo}
\renewcommand*{\lstlistlistingname}{Lista de \lstlistingname s} 
\renewcommand{\chaptername}{Capítulo}% Para cambiar el nombre por defecto'Chapter' por Capitulo
\renewcommand*\contentsname{Contenidos} %Para cambiar el nombre por defecto'Contents' por Contenidos
%----------------------------------------------------------------------------------------
%	TITLE PAGE
%----------------------------------------------------------------------------------------

\begingroup
\thispagestyle{empty}
\AddToShipoutPicture*{\put(6,5){\includegraphics[scale=1]{background}}} % Image background
\centering
\vspace*{9cm}
\par\normalfont\fontsize{35}{35}\sffamily\selectfont
Introducción a los algoritmos en c++\par % Book title
\vspace*{1cm}
{\Huge OscarGauss :P}\par % Author name
\endgroup

%----------------------------------------------------------------------------------------
%	COPYRIGHT PAGE
%----------------------------------------------------------------------------------------

\newpage
~\vfill
\thispagestyle{empty}

\noindent Copyright \copyright\ 2013 John Smith\\ % Copyright notice

\noindent \textsc{Published by Publisher}\\ % Publisher

\noindent \textsc{book-website.com}\\ % URL

\noindent Licensed under the Creative Commons Attribution-NonCommercial 3.0 Unported License (the ``License''). You may not use this file except in compliance with the License. You may obtain a copy of the License at \url{http://creativecommons.org/licenses/by-nc/3.0}. Unless required by applicable law or agreed to in writing, software distributed under the License is distributed on an \textsc{``as is'' basis, without warranties or conditions of any kind}, either express or implied. See the License for the specific language governing permissions and limitations under the License.\\ % License information

\noindent \textit{First printing, March 2013} % Printing/edition date

%----------------------------------------------------------------------------------------
%	TABLE OF CONTENTS
%----------------------------------------------------------------------------------------

%\chapterimage{chapter_head_000.pdf} % Table of contents heading image

\pagestyle{empty} % No headers

\tableofcontents % Print the table of contents itself

\cleardoublepage % Forces the first chapter to start on an odd page so it's on the right

\pagestyle{fancy} % Print headers again

%----------------------------------------------------------------------------------------
%	CHAPTERS
%----------------------------------------------------------------------------------------
%----------------------------------------------------------------------------------------
%	CHAPTER 1
%----------------------------------------------------------------------------------------

%\chapterimage{chapter_head_1.pdf} % Chapter heading image

\chapter{Introducción a la programación en c++}
\renewcommand{\chaptername}{Lecture}
\setlength{\parindent}{0pt}

\section{Introducción}\index{Introducción}

\subsection{¿Qué es un Lenguaje de Programación?}\index{Introducción!¿Qué es un Lenguaje de Programación?}

Antes de hablar de C++, es necesario explicar que un lenguaje de programación es una herramienta que nos permite comunicarnos e instruir a la computadora para que realice una tarea específica. Cada lenguaje de programación posee una sintaxis y un léxico particular, es decir, forma de escribirse que es diferente en cada uno por la forma que fue creado y por la forma que trabaja su compilador para revisar, acomodar y reservar el mismo programa en memoria.

Existen muchos lenguajes de programación de entre los que se destacan los siguientes:

\begin{itemize}
\item C
\item C++
\item Basic
\item Ada
\item Java
\item Pascal
\item Python
\item Fortran
\item Smalltalk
\end{itemize}

\subsection{Historia de C++}\index{Introducción!Historia de C++}

C++ es un lenguaje de programación creado por Bjarne Stroustrup en los laboratorios de At\&T en 1983. Stroustrup tomó como base el lenguaje de programación más popular en aquella época el cual era C.

El C++ es un derivado del mítico lenguaje C, el cual fue creado en la década de los 70 por la mano del finado Dennis Ritchie para la programación del sistema operativo (un sistema parecido a Unix es GNU/Linux), el cual surgió como un lenguaje orientado a la programación de sistemas (System Programming) y de herramientas (Utilities) recomendado sobre todo para programadores expertos, y que no llevaba implementadas muchas funciones que hacen a un lenguaje más comprensible.

Sin embargo, aunque esto en un inicio se puede convertir en un problema, en la práctica es su mayor virtud, ya que permite al programador un mayor control sobre lo que está haciendo. Años más tarde, un programador llamado Bjarne Stroustrup, creo lo que se conoce como C++.

Necesitaba ciertas facilidades de programación, incluidas en otros lenguajes pero que C no soportaba, al menos directamente, como son las llamadas clases y objetos, principios usados en la programación actual. Para ello rediseñó C, ampliando sus posibilidades pero manteniendo su mayor cualidad, la de permitir al programador en todo momento tener controlado lo que está haciendo, consiguiendo así una mayor rapidez que no se conseguiría en otros lenguajes.

C++ pretende llevar a C a un nuevo paradigma de clases y objetos con los que se realiza una comprensión más humana basándose en la construcción de objetos, con características propias solo de ellos, agrupados en clases. Es decir, si yo quisiera hacer un programa sobre animales, crearía una clase llamada animales, en la cual cada animal, por ejemplo un pato, sería un objeto, de tal manera que se ve el intento de esta forma de programar por ser un fiel reflejo de cómo los humanos (en teoría) manejamos la realidad.

Se dice que nuestro cerebro trabaja de forma relacional (relacionando hechos), es por ello que cada vez que recuerdas algo, (cuentas un hecho), termina siendo diferente (se agregan u omiten partes).

\subsection{¿Qué es C++?}\index{Introducción!¿Qué es C++?}

C++ es un lenguaje de programación orientado a objetos que toma la base del lenguaje C y le agrega la capacidad de abstraer tipos como en Smalltalk.

La intención de su creación fue el extender al exitoso lenguaje de programación C con mecanismos que permitieran la manipulación de objetos. En ese sentido, desde el punto de vista de los lenguajes orientados a objetos, el C++ es un lenguaje híbrido.

Posteriormente se añadieron facilidades de programación genérica, que se sumó a los otros dos paradigmas que ya estaban admitidos (programación estructurada y la programación orientada a objetos). Por esto se suele decir que el C++ es un lenguaje de programación multiparadigma.

\subsection{Herramientas Necesarias}\index{Introducción!Herramientas Necesarias}

Las principales herramientas necesarias para escribir un programa en C++ son las siguientes:

\begin{enumerate}
\item Un equipo ejecutando un sistema operativo.
\item Un compilador de C++
	\begin{itemize}
	\item Windows MingW (GCC para Windows) o MSVC (compilador de microsoft con versión gratuita)
	\item Linux (u otros UNIX): g++
	\item Mac (con el compilador Xcode)
	\end{itemize}
\item Un editor cualquiera de texto, o mejor un entorno de desarrollo (IDE)
	\begin{itemize}
	\item Windows:
		\begin{itemize}
		\item Microsoft Visual C++ (conocido por sus siglas MSVC). Incluye compilador y posee una versión gratuita (versión express)
		\item Bloc de notas (no recomendado)
		\item Editor Notepad++
		\item DevCpp (incluye MingW - en desuso, no recomendado, incluye también un compilador)
		\item Code::Blocks
		\end{itemize}
	\item Linux (o re-compilación en UNIX):
		\begin{itemize}
		\item Gedit
		\item Kate
		\item KDevelop
		\item Code::Blocks
		\item SciTE
		\item GVim
		\end{itemize}
	\item Mac:
		\begin{itemize}
		\item Xcode (con el compilador trae una IDE para poder programar)
		\end{itemize}
	\end{itemize}
\item Tiempo para practicar
\item Paciencia
\end{enumerate}

\textbf{Adicional}
\begin{itemize}
\item Inglés (Recomendado)
\item Estar familiarizado con C u otro lenguaje derivado (PHP, Python, etc).
\end{itemize}

Es recomendable tener conocimientos de C, debido a que C++ es una mejora de C, tener los conocimientos sobre este te permitira avanzar mas rapido y comprender aun mas. Tambien, hay que recordar que C++, admite C, por lo que se puede programar (reutilizar), funciones de C que se puedan usar en C++.

Aunque No es obligacion aprender C, es recomendable tener nociones sobre la programación orientada a objetos en el caso de no tener conocimientos previos de programación estructurada. Asimismo, muchos programadores recomiendan no saber C para saber C++, por ser el primero de ellos un lenguaje imperativo o procedimental y el segundo un lenguaje de programación orientado a objetos.

\subsection{Consejos iniciales antes de programar}\index{Introducción!Consejos iniciales antes de programar}

Con la práctica, se puede observar que se puede confundir a otros programadores con el código que se haga. Antes de siquiera hacer una línea de código, si se trabaja con otros programadores, ha de tenerse en cuenta que todos deben escribir de una forma similar el código, para que de forma global puedan corregir el código en el caso de que hubieran errores o rastrearlos en el caso de haberlos.

También es muy recomendable hacer uso de comentarios (comenta todo lo que puedas, hay veces que lo que parece obvio para ti, no lo es para los demás) y tratar de hacer un código limpio y comprensible, especificando detalles y haciendo tabulaciones, aunque te tome un poco mas de tiempo, es posible que mas adelante lo agradezcas tu mismo.

\subsection{Ejemplos}\index{Introducción!Ejemplos}

\begin{lstlisting}[style=Cpp,label=ejemplo-c++,caption=Ejemplo de C++]
#include <iostream>

using namespace std;

int main()
{
    int numero;
    cin>>numero;
    if(numero%2==0){
        cout<<"El numero es par\n";
    }else{
        cout<<"EL numero es impar\n";
    }
    return 0;
}
\end{lstlisting}


\section{Lo mas basico}\index{Lo mas basico}

\subsection{Proceso de desarrollo de un programa}\index{Lo mas basico!Proceso de desarrollo de un programa}

Si se desea escribir un programa en C++ se debe ejecutar como mínimo los siguientes pasos:
\begin{enumerate}
\item Escribir con un editor de texto plano un programa sintácticamente válido o usar un entorno de desarrollo (IDE) apropiado para tal fin
\item Compilar el programa y asegurarse de que no han habido errores de compilación
\item Ejecutar el programa y comprobar que no hay errores de ejecución
\end{enumerate}
Este último paso es el más costoso, por que en programas grandes, averiguar si hay o no un fallo prácticamente puede ser una tarea totémica.

Un archivo de C++ tiene la extención \textit{cpp} a continuación se escribe el siguiente codigo en c++ del archivo 'hola.cpp'

\begin{lstlisting}[style=Cpp, label=hola-mundo, caption=Hola Mundo]
// Aquí generalmente se suele indicar qué se quiere con el programa a hacer
// Programa que muestra 'Hola mundo' por pantalla y finaliza

// Aquí se sitúan todas las librerias que se vayan a usar con include,
// que se verá posteriormente
#include <iostream> // Esta libreria permite mostrar y leer datos por consola

int main()
{
    // Este tipo de líneas de código que comienzan por '//' son comentarios
    // El compilador los omite, y sirven para ayudar a otros programadores o
    // a uno mismo en caso de volver a revisar el código
    // Es una práctica sana poner comentarios donde se necesiten,

    std::cout << "Hola Mundo" << std::endl;

    // Mostrar por std::cout el mensaje Hola Mundo y comienza una nueva línea
    
    return 0;

    // se devuelve un 0.
    //que en este caso quiere decir que la salida se ha efectuado con éxito.
}
\end{lstlisting}

Mediante simple inspección, el código parece enorme, pero el compilador lo único que leerá para la creación del programa es lo siguiente:

\begin{lstlisting}[style=Cpp, label=hola-mundo-compilador, caption=Hola Mundo Compilado]
#include <iostream>
int main(void){ std::cout << "Hola Mundo" << std::endl; return 0; }
\end{lstlisting}

Como se puede observar, este código y el original no difieren en mucho salvo en los saltos de línea y que los comentarios, de los que se detallan posteriormente, están omitidos y tan sólo ha quedado "el esqueleto" del código legible para el compilador. Para el compilador, todo lo demás, sobra.

Aquí otro ejemplo

\begin{lstlisting}[style=Cpp, label=hello-world, caption=Hello World]
#include <iostream>

int main()
{
    std::cout << "Hola Mundo" << std::endl;
    std::cout << "Hello World" << std::endl;
    std::cout << "Hallo Welt" << std::endl;
    return 0;
}
\end{lstlisting}

Para hacer el código mas corto debemos incluir \textit{using namespace std} con lo que le estamos diciendo al compilador que usaremos el espacio de nombres std por lo que no tendremos que incluirlo cuando usemos elementos de este espacio de nombres, como pueden ser los objetos cout y cin, que representan el flujo de salida estándar (típicamente la pantalla o una ventana de texto) y el flujo de entrada estándar (típicamente el teclado).

Se veria de la siguiente manera:

\begin{lstlisting}[style=Cpp, label=using-namespace-std, caption=Using Namespace Std]
#include <iostream>

using namespace std;

int main()
{
    cout << "Hola Mundo" << endl;
    cout << "Hello World" << endl;
    cout << "Hallo Welt" << endl;
    return 0;
}
\end{lstlisting}

Los pasos siguientes son para una compilación en GNU o sistema operativo Unix, para generar el ejecutable del programa se compila con g++ de la siguiente forma:

\begin{listing}[style=consola, numbers=none,label=compilar,caption=Compilar]
| g++ hola.cpp -o hola
\end{listing}

Para poder ver los resultados del programa en acción, se ejecuta el programa de la siguiente forma:

\begin{listing}[style=consola, numbers=none,label=ejecutar,caption=Ejecutar]
|./hola
\end{listing}

Y a continuación se debe mostrar algo como lo siguiente:

\begin{listing}[style=consola, numbers=none,label=resultado,caption=Resultado]
Hola Mundo
\end{listing}

\subsection{Sintaxis}\index{Lo mas basico!Sintaxis}

Sintaxis es la forma correcta en que se deben escribir las instrucciones para el computador en un lenguaje de programación específico. C++ hereda la sintaxis de C estándar, es decir, la mayoría de programas escritos para el C estándar pueden ser compilados en C++.

\textbf{El punto y coma}

El punto y coma es uno de los simbólos más usados en C, C++; y se usa con el fin de indicar el final de una línea de instrucción. El punto y coma es de uso obligatorio.

\textbf{Ejemplo}

\begin{lstlisting}[style=Cpp, label=sintaxis, caption=Sintaxis]
clrscr(); //Limpiar pantalla, funciona con la librería conio de Borland C++
x = a + b;
 
string IP = "127.0.0.1"; // Variable IP tipo string
cout << IP << endl; // Devuelve 127.0.0.1
 
char Saludo[5] = "Hola"; // Variable Saludo tipo char
cout << Saludo[0] << endl; // Igual a H
cout << Saludo[1] << endl; // Igual a o
cout << Saludo[2] << endl; // Igual a l
cout << Saludo[3] << endl; // Igual a a
\end{lstlisting}

El punto y coma se usa también para separar contadores, condicionales e incrementadores dentro de un sentencia for

\textbf{Ejemplo}

\begin{lstlisting}[style=Cpp, label=sintaxis, caption=Sintaxis]
for (i=0; i < 10; i++) cout << i;
\end{lstlisting}

\textbf{Espacios y tabuladores}

Usar caracteres extras de espaciado o tabuladores ( caracteres tab ) es un mecanismo que nos permite ordenar de manera más clara el código del programa que estemos escribiendo, sin embargo, el uso de estos es opcional ya que el compilador ignora la presencia de los mismos. Por ejemplo, el segundo de los ejemplos anteriores se podría escribir como:

\begin{lstlisting}[style=Cpp, label=sintaxis, caption=Sintaxis]
for (int i=0; i < 10; i++) { cout << i * x; x++; }
\end{lstlisting}

y el compilador no pondría ningún reparo.


\subsection{Comentarios}\index{Lo mas basico!Comentarios}

Existen dos modos basicos para comentar en c++:
\begin{description}
\item [//] \hfill \\
Comentan solo una linea de código 
\item [/*Comentario*/] \hfill \\
Comentan estrofas de código
\end{description}

A continuación un ejemplo:

\begin{lstlisting}[style=Cpp, label=comentarios, caption=Comentarios]
#include <iostream>

using namespace std;

int main()
{
    /*
    cout    es para imprimir
    <<      se utiliza para separar elementos para cout
    endl    es un salto de linea
    */

    //En español
    cout << "Hola Mundo" << endl;
    //En ingles
    cout << "Hello World" << endl;
    //En aleman
    cout << "Hallo Welt" << endl;
    return 0;
}
\end{lstlisting}

\subsection{Datos primitivos}\index{Lo mas basico!Datos primitivos}

En un lenguaje de programación es indispensable poder almacenar información, para esto en C++ están disponibles los siguientes tipos que permiten almacenar información numérica de tipo entero o real:

\begin{center}
	\begin{tabular}{|c|c|c|c|}
		\hline
		Nombre & Descripción & Tamaño & Rango de valores\\
		\hline
		bool & Valor booleano & 1byte & true o false\\
		\hline
		char & Carácter o entero pequeño & 1byte & De -128 a 127\\
		\hline
		short int & Entero corto & 2bytes & De -32768 a 32767\\
		\hline
		int & Entero & 4bytes & De -2147483648 a 2147483647\\
		\hline
		long long & Entero largo & 8bytes & -9223372036854775808 a 9223372036854775807\\
		\hline
		float & Número de punto flotante & 4bytes & 3.4e +/- 38 (7 digitos)\\
		\hline
		double & Float con doble precisión & 8bytes & 1.7e +/- 308 (15 digitos)\\
		\hline
	\end{tabular}
\end{center}

Los valores dependen de la arquitectura utilizada. Los mostrados son los que generalmente se encuentran en una máquina típica de arquitectura 32 bits.

\textbf{El Modificador unsigned}

El modificador unsigned es utilizado únicamente con los enteros, su utilización permite utilizar en los enteros únicamente la parte positiva,

\begin{lstlisting}[style=Cpp, label=unsigned, caption=Unsigned]
int a; 			// Almacena valores entre -32768 y 32767
unsigned int a; // Almacena valores entre 0 y 65535
\end{lstlisting}


\subsection{Variables y constantes}\index{Lo mas basico!Variables y constantes}

Una variable, como su nombre lo indica, es un determinado objeto cuyo valor puede cambiar durante el proceso de una tarea específica. Contrario a una variable, una constante es un determinado objeto cuyo valor no puede ser alterado durante el proceso de una tarea específica. En C, C++ para declarar variables no existe una palabra especial, es decir, las variables se declarán escribiendo el tipo seguido de uno o más identificadores o nombres de variables. Por otro lado, para declarar constantes existe la palabra reservada const, así como la directiva \textit{\#define}. A continuación se muestran ejemplos de declaración de variables y constantes.
\begin{center}
	\begin{tabular}{|c|c|c|}
		\hline
		Variables & Constantes & Constantes\\
		\hline
		int a; & const int a = 100;	&  \#define a 100\\
		\hline
		float b; & const float b = 100; & \#define b 100\\
		\hline
	\end{tabular}
\end{center}

\textbf{Notas:}

A diferencia de las constantes declaradas con la palabra const los símbolos definidos con \textit{\#define} no ocupan espacio en la memoria del código ejecutable resultante.

El tipo de la variable o constante puede ser cualquiera de los listados en Tipos primitivos, o bien de un tipo definido por el usuario.

Las constantes son usadas a menudo con un doble propósito, el primero es con el fin de hacer más legible el código del programa, es decir, si se tiene (por ejemplo) la constante numerica 3.1416 y esta representa al número pi, entonces podemos hacer declaraciones tales como:

\begin{lstlisting}[style=Cpp, label=constantes, caption=Constantes]
#define pi 3.1416
\end{lstlisting}

En este caso podremos usar la palabra pi en cualquier parte del
programa y el compilador se encargará de cambiar dicho simbolo por 3.1416.
o bien,

\begin{lstlisting}[style=Cpp, label=constantes, caption=Constantes]
const pi = 3.1416;
\end{lstlisting}

En este otro caso podremos usar la palabra pi en cualquier parte del programa y el compilador se encargará
de cambiar dicho símbolo por una referencia a la constante pi guardada en la memoria.

\subsection{Lectura e Impresión}\index{Lo mas basico!Lectura e Impresión}

\textbf{La iostream}

La iostream es la biblioteca estándar en C++ para poder tener acceso a los dispositivos estándar de entrada y/o salida. En sus programas, si usted desea hacer uso de los objetos cin, cout, cerr y clog tendrá que incluir ( por medio de la directiva \textit{\#include} ) el uso de la biblioteca iostream. En la iostream se encuentran definidas las clases ios ( misma que es la base para las clases que implementen operaciones de entrada y/o salida de datos ), istream ( para operaciones de entrada ) y ostream ( para operaciones de salida ). Aparte de las clases mencionadas, en la iostream se encuentra una lista de variables y constantes ( atributos ) que son accesibles por el usuario a través del operador de ámbito ( :: ) si es que no se incluyera \textit{using namespace std}.

Si usted usa la directiva \#include <iostream.h> o \#include <iostream> en sus programas, automáticamente la iostream pone a su disposición los objetos cin, cout, clog y cerr en el ámbito estándar (std), de tal manera que usted puede comenzar a enviar o recibir información a través de los mismos sin siquiera preocuparse de su creación. Asi, un sencillo ejemplo del uso de los objetos mencionados se muestra en seguida.

\begin{lstlisting}[style=Cpp, label=iostream, caption=Iostream]
// De nuevo con el hola mundo...
#include <iostream.h>
int main()
{
    std::cout << "Hola mundo";    // imprimir mensaje (en la pantalla)
    std::cin.get();    // lectura ( entrada del teclado )
    return 0;
}
\end{lstlisting}

\textbf{Operadores de direccionamiento}

Los operadores de direccionamiento son los encargados de manipular el flujo de datos desde o hacia el dispositivo referenciado por un stream específico. El operador de direccionamiento para salidas es una pareja de símbolos de "menor que" <<, y el operador de direccionamiento para entradas es una pareja de símbolos de "mayor que" >>. Los operadores de direccionamiento se colocan entre dos operandos, el primero es el Stream y el segundo es una variable o constante que proporciona o recibe los datos de la operación. Por ejemplo, en el siguiente programa y en la instrucción cout << "Entre su nombre: "; la constante "Entre su nombre: " es la fuente o quien proporciona los datos para el objeto cout. Mientras que en la instrucción cin >> nombre la variable nombre es el destino o quien recibe los datos provenientes del objeto cin.

\begin{lstlisting}[style=Cpp, label=operdirec, caption=Oper Direc]
// De nuevo con el hola mundo...
#include <iostream.h>
int main()
{
    char nombre[80];
    cout << "Entre su nombre: ";
    cin  >> nombre;
    cout << "Hola," << nombre;
    cin.get();
    return 0;
}
\end{lstlisting}

Observe que si en una misma línea de comando se desea leer o escribir sobre varios campos a la vez, no es necesario nombrar más de una vez al stream. Ejemplos:

\begin{lstlisting}[style=Cpp, label=ejemplo-ioc, caption=Ejemplo Cin Cout]
cout << "Hola," << nombre;
cin >> A >> B >> C;
\end{lstlisting}

Otro ejemplo del manejo de \textit{cin}:

\begin{lstlisting}[style=Cpp, label=impresión, caption=Impresión]
// Programa que muestra diversos textos por consola

// Las librerías del sistema usadas son las siguientes
#include <iostream>

using namespace std;
// Es la función principal encargada de mostrar por consola diferentes textos
int main(void)
{
    // Ejemplo con una única línea, se muestra el uso de cout y endl
    cout << "Bienvenido. Soy un programa. Estoy en una linea de codigo." << endl;

    // Ejemplo con una única línea de código que se puede fraccionar
    // mediante el uso de '<<'
    cout << "Ahora "
         << "estoy fraccionado en el codigo, pero en la consola me muestro como una unica frase."
         << endl;

    // Uso de un código largo, que cuesta leer para un programador,
    // y que se ejecutará sin problemas.
    // *** No se recomienda hacer líneas de esta manera,
    // esta forma de programar no es apropiada ***
    cout << "Un gran texto puede ocupar muchas lineas."
         << endl
         << "Pero eso no frena al programador a que todo se pueda poner en una unica linea de codigo y que"
         << endl
         << "el programa, al ejecutarse, lo situe como el programador quiso"
         << endl;

    return 0; // Y se termina con éxito.
}
\end{lstlisting}

Consola:

\begin{listing}[style=consola, numbers=none,label=resultado-impresión,caption=Resultado Impresión]
Bienvenido. Soy un programa. Estoy en una linea de codigo.
Ahora estoy fraccionado en el codigo, pero en la consola me muestro como una unica frase.
Un gran texto puede ocupar muchas lineas.
Pero eso no frena al programador a que todo se pueda poner en una unica linea de codigo y que
el programa, al ejecutarse, lo situe como el programador quiso

\end{listing}


\subsection{Operadores}\index{Lo mas basico!Operadores}

El C++ está lleno de operadores. Presentamos aquí una tabla de los mismos indicando el rango de prioridad de cada uno, y cómo se ejecutan. A continuación comentaremos brevemente los operadores.

\begin{center}
	\begin{tabular}{|c|c|}
		\hline
		\textbf{Operadores (de mayor a menor prioridad)} & \textbf{Sentido} \\
		\hline
		( ) \{ \} & I-D \\
		\hline
		! ++ --  * \& & D-I \\
		\hline
		* / \% & I-D \\
		\hline
		+ - & I-D \\
		\hline
		<< >> & I-D \\
		\hline
		< <= > >= & I-D \\
		\hline
		== != & I-D \\
		\hline
		\& & I-D \\
		\hline
		\textasciicircum & I-D \\
		\hline
		| & I-D \\
		\hline
		\&\& (and) & I-D \\
		\hline
		|| (or) & I-D \\
		\hline
		?: (Operador Condicional Ternario) & I-D \\
		\hline
		= += -= *= /= \%= & D-I \\
		\hline
	\end{tabular}
\end{center}

La acción de estos operadores es la siguiente:

\subsubsection{Operadores Aritméticos}\index{Lo mas basico!Operadores!Operadores Aritméticos}

\begin{description}
\item [+] Suma los valores situados a su derecha y a su izquierda.
\item [-] Resta el valor de su derecha del valor de su izquierda.
\item [-] Como operador unario, cambia el signo del valor  de su izquierda.
\item [*] Multiplica el valor de su derecha por el valor de su izquierda.
\item [/] Divide el valor situado a su izquierda por el valor situado a su dercha.
\item [\%] Proporciona el resto de la división del valor de la izquierda por el valor de la derecha (sólo enteros).
\item [++] Suma $1$ al valor de la variable situada a su izquierda (modo prefijo) o de la variable situada a su derecha (modo sufijo).
\item [--] Igual que ++, pero restando $1$. 
\end{description}

\subsubsection{Operadores de Asignación}\index{Lo mas basico!Operadores!Operadores de Asignación}

\begin{description}
\item [=] Asigna el valor de su derecha a la variable de su izquierda.
\end{description}

Cada uno de los siguientes operadores actualiza la variable de su izquierda con el valor de su derecha utilizando la operación indicada. Usaremos de \textit{d} e \textit{i} para iquierda.

\begin{description}
\item [+=] Suma la cantidad \textit{d} a la variable \textit{i}.
\item [-=] Resta la cantidad \textit{d} de la variable \textit{i}.
\item [*=] Multiplica la variable \textit{i} por la variable \textit{d}.
\item [/=] Divide la variable \textit{i} entre la cantidad \textit{d}.
\item [\%=] Proporciona el resto de la división de la variable \textit{i}  por la cantidad \textit{d}
\end{description}

\textbf{Ejemplo}

conejos *= 1.6; //es lo mismo que conejos = conejos * 1.6;

\subsubsection{Operadores de Relación}\index{Lo mas basico!Operadores!Operadores de Relación}

Cada uno de estos operadores compara el valor de su izquierda con el valor de su dercha. La expresión de relación formada por un operador y sus dos operandos toma el valor $1$ si la expresión es cierta, y el valor $0$ si es falsa.

\begin{description}
\item [<] menor que
\item [<=] menor o igual que 
\item [==] igual a
\item [>=] mayor o igual que
\item [>] mayor que 
\item [!=] distinto de
\end{description}

\subsubsection{Operadores Lógicos}\index{Lo mas basico!Operadores!Operadores Lógicos}

Los operadores lógicos utilizan normalmenteexpresiones de relación como operadores. El operador \textit{!} toma un operando situado a su derecha; el resto toma dos: uno a su derecha y otro a su izquierda.

\begin{description}
\item [\&\& and] La expresión combinada es cierta si ambos operandos lo son, y falsa en cualquier otro caso.
\item [|| or] La expresión combinada es cierta si uno o ambos operandos lo son, y falsa en cualquier otro caso.
\item [! not] La expresión es cierta si el operador es falso, y viceversa.
\end{description}

\subsubsection{Operadores Relacionados con punteros}\index{Lo mas basico!Operadores!Operadores Relacionados con punteros}

\begin{description}
\item [\& Operador de Dirección] Cuando va seguido por el nombre de una variable, entrega la dirección de dicha variable \textit{\&abc} es la dirección de la variable \textit{abc}.
\item [* Operador de Indirección] Cuando va seguido por un puntero, entrega el valor almacenado en la dirección apuntada por él \textit{abc = 22; def = \&abc; val = *def} el efecto es asignar a \textit{val} el valor de 22.
\end{description}

\subsubsection{Operadores de Estructuras y Uniones}\index{Lo mas basico!Operadores!Operadores de Estructuras y Uniones}

\subsubsection{Operadores Lógicos de Bits}\index{Lo mas basico!Operadores!Operadores Lógicos de Bits}
\subsubsection{Misceláneas}\index{Lo mas basico!Operadores!Misceláneas}







\section{Estructuras de Control}\index{Estructuras de Control}

Las estructuras de control permiten modificar el flujo de ejecución de las instrucciones de un programa.

Con las estructuras de control se puede:

\begin{itemize}
\item De acuerdo a una condición, ejecutar un grupo u otro de sentencias (\textbf{If})
\item De acuerdo al valor de una variable, ejecutar un grupo u otro de sentencias (\textbf{Switch})
\item Ejecutar un grupo de sentencias mientras se cumpla una condición (\textbf{While})
\item Ejecutar un grupo de sentencias hasta que se cumpla una condición (\textbf{Do-While})
\item Ejecutar un grupo de sentencias un número determinado de veces (\textbf{For})
\end{itemize}

Todas las estructuras de control tienen un único punto de entrada y un único punto de salida.


\subsection{Sentencias de decisión}\index{Estructuras de Control!Sentencias de decisión}

\begin{center}
	\textbf{Definición}
\end{center}

Las sentencias de decisión o también llamadas de CONTROL DE FLUJO son estructuras de control que realizan una pregunta la cual retorna verdadero o falso (evalúa una condicion) y selecciona la siguiente instrucción a ejecutar dependiendo la respuesta o resultado.


\subsubsection{Sentencia if}\index{Estructuras de Control!Sentencias de decisión!Sentencia if}

La instrucción if es, por excelencia, la más utilizada para construir estructuras de control de flujo.

\begin{center}
	\textbf{Sintaxis}
\end{center}

\textbf{Primera Forma}

Ahora bién, la sintaxis utilizada en la programación de C++ es la siguiente:

\begin{lstlisting}[style=Cpp, label=sintaxis-if, caption=Sintaxis If]
if (condicion)
   {
       Set de instrucciones
   }
\end{lstlisting}

siendo \textit{"condicion"} el lugar donde se pondrá la condición que se tiene que cumplir para que sea verdadera la sentencia y así proceder a realizar el \textit{"set de instrucciones"} o código contenido dentro de la sentencia.

\textbf{Segunda Forma}

Ahora veremos la misma sintaxis pero ahora le añadiremos la parte \textit{"Falsa"} de la sentencia:

\begin{lstlisting}[style=Cpp, label=sintaxis-if-else, caption=Sintaxis If Else]
if (condicion)
   {
       Set de instrucciones    //Parte VERDADERA
   }
else
   {
       Set de instrucciones 2  //Parte FALSA
   }
\end{lstlisting}

La forma mostrada anteriormente muestra la union de la parte \textit{"VERDADERA"} con la nueva secuencia la cual es la parte \textit{"FALSA"} de la sentencia de decision \textit{"IF"} en la cual esta compuesta por el:

\begin{lstlisting}[style=Cpp, label=sintaxis-else, caption=Sintaxis Else]
else
   {
     Set de instrucciones 2  //Parte FALSA
   }
\end{lstlisting}

la palabra \textit{"else"} o \textit{"De lo contrario"} indica al lenguaje que de lo contrario al no ser verdadera o no se cumpla la parte verdadera entonces realizara el \textit{"set de instrucciones 2"}.

\begin{center}
	\textbf{Ejemplos De Sentencias If}
\end{center}

\textbf{Ejemplo 1:}

\begin{lstlisting}[style=Cpp, label=if-ejemplo-1, caption=If Ejemplo 1]
if(numero == 0) //La condicion indica que tiene que ser igual a Cero
    {
      cout<<"El Numero Ingresado es Igual a Cero";
    }
\end{lstlisting}

\textbf{Ejemplo 2:}

\begin{lstlisting}[style=Cpp, label=if-ejemplo-2, caption=If Ejemplo 2]
if(numero > 0) // la condicion indica que tiene que ser mayor a Cero
    {    
      cout<<"El Numero Ingresado es Mayor a Cero";
    }
\end{lstlisting}

\textbf{Ejemplo 3:}

\begin{lstlisting}[style=Cpp, label=if-ejemplo-3, caption=If Ejemplo 3]
if(numero < 0) // la condicion indica que tiene que ser menor a Cero
    {    
      cout<<"El Numero Ingresado es Menor a Cero";
    }
\end{lstlisting}

Ahora uniremos todos estos ejemplos para formar un solo programa mediante la utilización de la sentencia "Else" e introduciremos el hecho de que se puede escribir en este espacio una sentencia if ya que podemos ingresar cualquier tipo de código dentro de la sentencia escrita después de un Else.

\textbf{Ejemplo 4:}

\begin{lstlisting}[style=Cpp, label=if-ejemplo-4, caption=If Ejemplo 4]
if(numero == 0) //La condicion indica que tiene que ser igual a Cero
   {
     cout<<"El Numero Ingresado es Igual a Cero";
   }
else
   {
      if(numero > 0) // la condicion indica que tiene que ser mayor a Cero
      {    
          cout<<"El Numero Ingresado es Mayor a Cero";
      }
      else
      {
         if(numero < 0) // la condicion indica que tiene que ser menor a Cero
         {    
           cout<<"El Numero Ingresado es Menor a Cero";
         }
      }
    }
\end{lstlisting}


\subsubsection{Sentencia switch}\index{Estructuras de Control!Sentencias de decisión!Sentencia switch}

switch es otra de las instrucciones que permiten la construcción de estructuras de control. A diferencia de if, para controlar el flujo por medio de una sentencia switch se debe de combinar con el uso de las sentencias \textit{case} y \textit{break}.

\textbf{Notas:} cualquier número de casos a evaluar por switch así como la sentencia default son opcionales. La sentencia switch es muy útil en los casos de presentación de menus.

\begin{center}
	\textbf{Sintaxis}
\end{center}
Ahora bién, la sintaxis utilizada en la programación de C++ es la siguiente:

\begin{lstlisting}[style=Cpp, label=sintaxis-switch, caption=Sintaxis Switch]
switch (condicion)
{
    case primer_caso:
         bloque de instrucciones 1
    break;

    case segundo_caso:
         bloque de instrucciones 2
    break;

    case caso_n:
         bloque de instrucciones n
    break;

    default: bloque de instrucciones por defecto
}
\end{lstlisting}

\begin{center}
	\textbf{Ejemplos De Sentencias Switch}
\end{center}

\textbf{Ejemplo 1}

\begin{lstlisting}[style=Cpp, label=switch-ejemplo-1, caption=Switch Ejemplo 1]
switch (numero)
{
    case 0: cout << "numero es cero";
}
\end{lstlisting}

\textbf{Ejemplo 2}

\begin{lstlisting}[style=Cpp, label=switch-ejemplo-2, caption=Switch Ejemplo 2]
switch (opcion)
{
    case 0: cout << "Su opcion es cero"; break;
    case 1: cout << "Su opcion es uno"; break;
    case 2: cout << "Su opcion es dos";
}
\end{lstlisting}

\textbf{Ejemplo 3}

\begin{lstlisting}[style=Cpp, label=switch-ejemplo-3, caption=Switch Ejemplo 3]
switch (opcion)
{
    case 1: cout << "Su opcion es 1"; break;
    case 2: cout << "Su opcion es 2"; break;
    case 3: cout << "Su opcion es 3"; break;
    default: cout << "Elija una opcion entre 1 y 3";
}
\end{lstlisting}


\subsubsection{Operador condicional ternario ?:}\index{Estructuras de Control!Sentencias de decisión!Operador condicional ternario ?:}

En C/C++, existe el operador condicional ( ?: ) el cual es conocido por su estructura como ternario. El comportamiento de dicho operador es el mismo que una estructura if - then - else del lenguaje BASIC (y de la función IIf de Visual Basic). El operador condicional ?: es útil para evaluar situaciones tales como:

Si se cumple tal condición entonces haz esto, de lo contrario haz esto otro.

\begin{center}
	\textbf{Sintaxis}
\end{center}

\begin{lstlisting}[style=Cpp, label=sintaxis-switch, caption=Sintaxis Switch]
( (condicion) ? proceso1 : proceso2 )
\end{lstlisting}

En donde, condicion es la expresión que se evalua, proceso1 es la tarea a realizar en el caso de que la evaluación resulte verdadera, y proceso2 es la tarea a realizar en el caso de que la evaluación resulte falsa.

\begin{center}
	\textbf{Ejemplos De Sentencias Operador condicional ternario ?:}
\end{center}

\textbf{Ejemplo 1}

\begin{lstlisting}[style=Cpp, label=condicional-ejemplo-1, caption=Condicional Ejemplo 1]
int edad;
cout << "Cual es tu edad: ";
cin >> edad;
cout << ( (edad < 18) ? "Eres joven aun" : "Ya tienes la mayoria de edad" );
El ejemplo anterior podria escribirse de la siguiente manera:

int edad;
cout << "Cual es tu edad: ";
cin >> edad;
if (edad < 18) cout << "Eres joven aun";
    else cout << "Ya tienes la mayoria de edad";
\end{lstlisting}

\textbf{Ejemplo 2}

Vamos a suponer que deseamos escribir una función que opere sobre dos valores numéricos y que la misma ha de regresar 1 (true) en caso de que el primer valor pasado sea igual al segundo valor; en caso contrario la función debe retornar 0 (false).

\begin{lstlisting}[style=Cpp, label=condicional-ejemplo-2, caption=Condicional Ejemplo 2]
int es_igual( int a, int b)
{
    return ( (a == b) ? 1 : 0 )
}
\end{lstlisting}

\subsection{Sentencias de iteración}\index{Estructuras de Control!Sentencias de iteración}

\begin{center}
	\textbf{Definición}
\end{center}

Las Sentencias de Iteración o Ciclos son estructuras de control que repiten la ejecución de un grupo de instrucciones. Básicamente, una sentencia de iteración es una estructura de control condicional, ya que dentro de la misma se repite la ejecución de una o más instrucciones mientras que una a condición especifica se cumpla. Muchas veces tenemos que repetir un número definido o indefinido de veces un grupo de instrucciones por lo que en estos casos utilizamos este tipo de sentencias. en C++ los ciclos o bucles se construyen por medio de las sentencias for, while y do - while. La sentencia for es útil para los casos en donde se conoce de antemano el número de veces que una o más sentencias han de repetirse. Por otro lado, la sentencia while es útil en aquellos casos en donde no se conoce de antemano el número de veces que una o más sentencias se tienen que repetir.

\subsubsection{Sentencias For}\index{Estructuras de Control!Sentencias de iteración!Sentencias For}

Ahora bién, la sintaxis utilizada en la programación de C++ es la siguiente:

\begin{center}
	\textbf{Sintaxis}
\end{center}

\begin{lstlisting}[style=Cpp, label=sintaxis-for, caption=Sintaxis For]
for(contador; final; incremento)
{
    Codigo a Repetir;
}
\end{lstlisting}

donde:

\begin{description}
\item [contador] es una variable numérica
\item [final] es la condición que se evalua para finalizar el ciclo (puede ser independiente del contador)
\item [incremento] es el valor que se suma o resta al contador
\end{description}

Hay que tener en cuenta que el \textit{"for"} evalua la condición de finalización igual que el while, es decir, mientras esta se cumpla continuaran las repeticiones.

\begin{center}
	\textbf{Ejemplos De Sentencias For}
\end{center}

\textbf{Ejemplo 1}

\begin{lstlisting}[style=Cpp, label=for-ejemplo-1, caption=For Ejemplo 1]
for(int i=1; i<=10; i++)
{
    cout<<"Hola Mundo";   
}
\end{lstlisting}

Esto indica que el contador \textit{"i"} inicia desde 1 y continuará iterando mientras i sea menor o igual a 10 ( en este caso llegará hasta 10) e \textit{"i++"} realiza la sumatoria por unidad lo que hace que el for y el contador se sumen. repitiendo 10 veces \textit{"HOLA MUNDO"} en pantalla.

\textbf{Ejemplo 2}

\begin{lstlisting}[style=Cpp, label=for-ejemplo-2, caption=For Ejemplo 2]
for(int i=10; i>=0; i--)
{
    cout<<"Hola Mundo";   
}
\end{lstlisting}

Este ejemplo hace lo mismo que el primero, salvo que el contador se inicializa a 10 en lugar de 1; y por ello cambia la condición que se evalua así como que el contador se decrementa en lugar de ser incrementado.

La condición también puede ser independiente del contador:

\textbf{Ejemplo 3}

\begin{lstlisting}[style=Cpp, label=for-ejemplo-3, caption=For Ejemplo 3]
int j = 20;
for(int i=0; j>0; i++){
    cout<<"Hola"<<i<<" - "<<j<<endl;
    j--;
}
\end{lstlisting}

En este ejemplo las iteraciones continuaran mientras j sea mayor que 0, sin tener en cuenta el valor que pueda tener i.


\subsubsection{Sentencia While}\index{Estructuras de Control!Sentencias de iteración!Sentencia While}

Ahora bién, la sintaxis utilizada en la programación de C++ es la siguiente:

\begin{center}
	\textbf{Sintaxis}
\end{center}

\begin{lstlisting}[style=Cpp, label=sintaxis-while, caption=Sintaxis While]
while(condicion)
{
    codigo a Repetir
}
\end{lstlisting}

donde:

\begin{description}
\item [condicion] es la expresión a evaluar
\end{description}

\begin{center}
	\textbf{Ejemplos De Sentencias While}
\end{center}

\textbf{Ejemplo 1}

\begin{lstlisting}[style=Cpp, label=while-ejemplo-1, caption=While Ejemplo 1]
int contador = 0;
 
while(contador<=10)
{
    contador=contador+1;
    cout<<"Hola Mundo";
}
\end{lstlisting}

El contador Indica que hasta que este llegue a el total de 10 entonces se detendrá y ya no se realizará el código contenido dentro de la sentencia while, de lo contrario mientras el \textit{"contador"} sea menor a 10 entonces el código contenido se ejecutará desplegando hasta 10 veces "Hola Mundo" en pantalla.


\subsubsection{Sentencia Do - While}\index{Estructuras de Control!Sentencias de iteración!Sentencia Do - While}

\begin{center}
	\textbf{Sintaxis}
\end{center}

La sentencia do es usada generalmente en cooperación con while para garantizar que una o más instrucciones se ejecuten al menos una vez. Por ejemplo, en la siguiente construcción no se ejecuta nada dentro del ciclo while, el hecho es que el contador inicialmente vale cero y la condición para que se ejecute lo que está dentro del while es \textit{"mientras el contador sea mayor que diez"}. Es evidente que a la primera evaluación hecha por while la condición deja de cumplirse.

\begin{lstlisting}[style=Cpp, label=sintaxis-do-while, caption=Sintaxis Do While]
int contador = 0;
 
while(contador > 10)
{
    contador ++;
    cout<<"Hola Mundo";
}
\end{lstlisting}

Al modificar el segmento de código anterior usando do tenemos:

\begin{lstlisting}[style=Cpp, label=sintaxis-do-while, caption=Sintaxis Do While]
int contador = 0;
 
do
{
    contador ++;
    cout<<"Hola Mundo";
}
while(contador > 10);
\end{lstlisting}

Observe cómo en el caso de do la condición es evaluada al final en lugar de al principio del bloque de instrucciones y, por lo tanto, el código que le sigue al \textit{do} se ejecuta al menos la primera vez.


\subsection{Sentencias Break y Continue}\index{Estructuras de Control!Sentencias Break y Continue}

En la sección (Sentencia switch) vimos que la sentencia break es utilizada con el propósito de forzar un salto dentro del bloque switch hacia el final del mismo. En esta sección volveremos a ver el uso de break, salvo que esta ocasión la usaremos junto con las sentecias for y la sentencia while. Además, veremos el uso de la sentencia continue.


\subsubsection{Break}\index{Estructuras de Control!Sentencias de iteración!Sentencias Break y Continue!Break}

La sentencia break se usa para forzar un salto hacia el final de un ciclo controlado por for o por while.

\begin{center}
	\textbf{Ejemplo}
\end{center}

En el siguiente fragmento de código la sentencia break cierra el ciclo for cuando la variable ( i ) es igual a 5. 

\begin{lstlisting}[style=Cpp, label=break, caption=Break]
for (int i=0; i<10; i++) {
    if (i == 5) break;
    cout << i << " ";
}
\end{lstlisting}

La salida para el mismo será:

\begin{listing}[style=consola, numbers=none,label=resultado-break,caption=Resultado Break]
0 1 2 3 4
\end{listing}


\subsubsection{Continue}\index{Estructuras de Control!Sentencias de iteración!Sentencias Break y Continue!Continue}

La sentencia continue se usa para ignorar una iteración dentro de un ciclo controlado por for o por while.

\begin{center}
	\textbf{Ejemplo}
\end{center}

En el siguiente fragmento de código la sentencia continue ignora la iteración cuando la variable ( i ) es igual a 5.

\begin{lstlisting}[style=Cpp, label=continue, caption=Continue]
for (int i=0; i<10; i++) {
    if (i == 5) continue;
    cout << i << " ";
}
\end{lstlisting}

La salida para el mismo será:

\begin{listing}[style=consola, numbers=none,label=resultado-continue,caption=Resultado Continue]
0 1 2 3 4 6 7 8 9
\end{listing}

\subsubsection{Uso de break y continue junto con while}\index{Estructuras de Control!Sentencias de iteración!Sentencias Break y Continue!Uso de break y continue junto con while}

Los dos ejemplos anteriores se presentan en seguida, salvo que en lugar de for se hace uso de while.

\textbf{Nota:} no deje de observar que la construcción del ciclo while para el caso de la sentencia continue es diferente, esto para garantizar que el ciclo no vaya a caer en una iteración infinita.

\begin{center}
	\textbf{Break}
\end{center}

\begin{lstlisting}[style=Cpp, label=break-while, caption=Break While]
int i = 0;
while (i<10) {
    if (i == 5) break;
    cout << i << " ";
    i++;
}
\end{lstlisting}

\begin{center}
	\textbf{Continue}
\end{center}

\begin{lstlisting}[style=Cpp, label=continue-while, caption=Continue While]
int i = -1;
while (i<10) {
    i++;
    if (i == 5) continue;
    cout << i << " ";
}
\end{lstlisting}

%----------------------------------------------------------------------------------------
%	CHAPTER 1
%----------------------------------------------------------------------------------------

%\chapterimage{chapter_head_1.pdf} % Chapter heading image

\chapter{Introducción a la programación en c++}
\renewcommand{\chaptername}{Lecture}
\setlength{\parindent}{0pt}

\section{Introducción}\index{Introducción}

\subsection{¿Qué es un Lenguaje de Programación?}\index{Introducción!¿Qué es un Lenguaje de Programación?}

Antes de hablar de C++, es necesario explicar que un lenguaje de programación es una herramienta que nos permite comunicarnos e instruir a la computadora para que realice una tarea específica. Cada lenguaje de programación posee una sintaxis y un léxico particular, es decir, forma de escribirse que es diferente en cada uno por la forma que fue creado y por la forma que trabaja su compilador para revisar, acomodar y reservar el mismo programa en memoria.

Existen muchos lenguajes de programación de entre los que se destacan los siguientes:

\begin{itemize}
\item C
\item C++
\item Basic
\item Ada
\item Java
\item Pascal
\item Python
\item Fortran
\item Smalltalk
\end{itemize}

\subsection{Historia de C++}\index{Introducción!Historia de C++}

C++ es un lenguaje de programación creado por Bjarne Stroustrup en los laboratorios de At\&T en 1983. Stroustrup tomó como base el lenguaje de programación más popular en aquella época el cual era C.

El C++ es un derivado del mítico lenguaje C, el cual fue creado en la década de los 70 por la mano del finado Dennis Ritchie para la programación del sistema operativo (un sistema parecido a Unix es GNU/Linux), el cual surgió como un lenguaje orientado a la programación de sistemas (System Programming) y de herramientas (Utilities) recomendado sobre todo para programadores expertos, y que no llevaba implementadas muchas funciones que hacen a un lenguaje más comprensible.

Sin embargo, aunque esto en un inicio se puede convertir en un problema, en la práctica es su mayor virtud, ya que permite al programador un mayor control sobre lo que está haciendo. Años más tarde, un programador llamado Bjarne Stroustrup, creo lo que se conoce como C++.

Necesitaba ciertas facilidades de programación, incluidas en otros lenguajes pero que C no soportaba, al menos directamente, como son las llamadas clases y objetos, principios usados en la programación actual. Para ello rediseñó C, ampliando sus posibilidades pero manteniendo su mayor cualidad, la de permitir al programador en todo momento tener controlado lo que está haciendo, consiguiendo así una mayor rapidez que no se conseguiría en otros lenguajes.

C++ pretende llevar a C a un nuevo paradigma de clases y objetos con los que se realiza una comprensión más humana basándose en la construcción de objetos, con características propias solo de ellos, agrupados en clases. Es decir, si yo quisiera hacer un programa sobre animales, crearía una clase llamada animales, en la cual cada animal, por ejemplo un pato, sería un objeto, de tal manera que se ve el intento de esta forma de programar por ser un fiel reflejo de cómo los humanos (en teoría) manejamos la realidad.

Se dice que nuestro cerebro trabaja de forma relacional (relacionando hechos), es por ello que cada vez que recuerdas algo, (cuentas un hecho), termina siendo diferente (se agregan u omiten partes).

\subsection{¿Qué es C++?}\index{Introducción!¿Qué es C++?}

C++ es un lenguaje de programación orientado a objetos que toma la base del lenguaje C y le agrega la capacidad de abstraer tipos como en Smalltalk.

La intención de su creación fue el extender al exitoso lenguaje de programación C con mecanismos que permitieran la manipulación de objetos. En ese sentido, desde el punto de vista de los lenguajes orientados a objetos, el C++ es un lenguaje híbrido.

Posteriormente se añadieron facilidades de programación genérica, que se sumó a los otros dos paradigmas que ya estaban admitidos (programación estructurada y la programación orientada a objetos). Por esto se suele decir que el C++ es un lenguaje de programación multiparadigma.

\subsection{Herramientas Necesarias}\index{Introducción!Herramientas Necesarias}

Las principales herramientas necesarias para escribir un programa en C++ son las siguientes:

\begin{enumerate}
\item Un equipo ejecutando un sistema operativo.
\item Un compilador de C++
	\begin{itemize}
	\item Windows MingW (GCC para Windows) o MSVC (compilador de microsoft con versión gratuita)
	\item Linux (u otros UNIX): g++
	\item Mac (con el compilador Xcode)
	\end{itemize}
\item Un editor cualquiera de texto, o mejor un entorno de desarrollo (IDE)
	\begin{itemize}
	\item Windows:
		\begin{itemize}
		\item Microsoft Visual C++ (conocido por sus siglas MSVC). Incluye compilador y posee una versión gratuita (versión express)
		\item Bloc de notas (no recomendado)
		\item Editor Notepad++
		\item DevCpp (incluye MingW - en desuso, no recomendado, incluye también un compilador)
		\item Code::Blocks
		\end{itemize}
	\item Linux (o re-compilación en UNIX):
		\begin{itemize}
		\item Gedit
		\item Kate
		\item KDevelop
		\item Code::Blocks
		\item SciTE
		\item GVim
		\end{itemize}
	\item Mac:
		\begin{itemize}
		\item Xcode (con el compilador trae una IDE para poder programar)
		\end{itemize}
	\end{itemize}
\item Tiempo para practicar
\item Paciencia
\end{enumerate}

\textbf{Adicional}
\begin{itemize}
\item Inglés (Recomendado)
\item Estar familiarizado con C u otro lenguaje derivado (PHP, Python, etc).
\end{itemize}

Es recomendable tener conocimientos de C, debido a que C++ es una mejora de C, tener los conocimientos sobre este te permitira avanzar mas rapido y comprender aun mas. Tambien, hay que recordar que C++, admite C, por lo que se puede programar (reutilizar), funciones de C que se puedan usar en C++.

Aunque No es obligacion aprender C, es recomendable tener nociones sobre la programación orientada a objetos en el caso de no tener conocimientos previos de programación estructurada. Asimismo, muchos programadores recomiendan no saber C para saber C++, por ser el primero de ellos un lenguaje imperativo o procedimental y el segundo un lenguaje de programación orientado a objetos.

\subsection{Consejos iniciales antes de programar}\index{Introducción!Consejos iniciales antes de programar}

Con la práctica, se puede observar que se puede confundir a otros programadores con el código que se haga. Antes de siquiera hacer una línea de código, si se trabaja con otros programadores, ha de tenerse en cuenta que todos deben escribir de una forma similar el código, para que de forma global puedan corregir el código en el caso de que hubieran errores o rastrearlos en el caso de haberlos.

También es muy recomendable hacer uso de comentarios (comenta todo lo que puedas, hay veces que lo que parece obvio para ti, no lo es para los demás) y tratar de hacer un código limpio y comprensible, especificando detalles y haciendo tabulaciones, aunque te tome un poco mas de tiempo, es posible que mas adelante lo agradezcas tu mismo.

\subsection{Ejemplos}\index{Introducción!Ejemplos}

\begin{lstlisting}[style=Cpp,label=ejemplo-c++,caption=Ejemplo de C++]
#include <iostream>

using namespace std;

int main()
{
    int numero;
    cin>>numero;
    if(numero%2==0){
        cout<<"El numero es par\n";
    }else{
        cout<<"EL numero es impar\n";
    }
    return 0;
}
\end{lstlisting}


\section{Lo mas basico}\index{Lo mas basico}

\subsection{Proceso de desarrollo de un programa}\index{Lo mas basico!Proceso de desarrollo de un programa}

Si se desea escribir un programa en C++ se debe ejecutar como mínimo los siguientes pasos:
\begin{enumerate}
\item Escribir con un editor de texto plano un programa sintácticamente válido o usar un entorno de desarrollo (IDE) apropiado para tal fin
\item Compilar el programa y asegurarse de que no han habido errores de compilación
\item Ejecutar el programa y comprobar que no hay errores de ejecución
\end{enumerate}
Este último paso es el más costoso, por que en programas grandes, averiguar si hay o no un fallo prácticamente puede ser una tarea totémica.

Un archivo de C++ tiene la extención \textit{cpp} a continuación se escribe el siguiente codigo en c++ del archivo 'hola.cpp'

\begin{lstlisting}[style=Cpp, label=hola-mundo, caption=Hola Mundo]
// Aquí generalmente se suele indicar qué se quiere con el programa a hacer
// Programa que muestra 'Hola mundo' por pantalla y finaliza

// Aquí se sitúan todas las librerias que se vayan a usar con include,
// que se verá posteriormente
#include <iostream> // Esta libreria permite mostrar y leer datos por consola

int main()
{
    // Este tipo de líneas de código que comienzan por '//' son comentarios
    // El compilador los omite, y sirven para ayudar a otros programadores o
    // a uno mismo en caso de volver a revisar el código
    // Es una práctica sana poner comentarios donde se necesiten,

    std::cout << "Hola Mundo" << std::endl;

    // Mostrar por std::cout el mensaje Hola Mundo y comienza una nueva línea
    
    return 0;

    // se devuelve un 0.
    //que en este caso quiere decir que la salida se ha efectuado con éxito.
}
\end{lstlisting}

Mediante simple inspección, el código parece enorme, pero el compilador lo único que leerá para la creación del programa es lo siguiente:

\begin{lstlisting}[style=Cpp, label=hola-mundo-compilador, caption=Hola Mundo Compilado]
#include <iostream>
int main(void){ std::cout << "Hola Mundo" << std::endl; return 0; }
\end{lstlisting}

Como se puede observar, este código y el original no difieren en mucho salvo en los saltos de línea y que los comentarios, de los que se detallan posteriormente, están omitidos y tan sólo ha quedado "el esqueleto" del código legible para el compilador. Para el compilador, todo lo demás, sobra.

Aquí otro ejemplo

\begin{lstlisting}[style=Cpp, label=hello-world, caption=Hello World]
#include <iostream>

int main()
{
    std::cout << "Hola Mundo" << std::endl;
    std::cout << "Hello World" << std::endl;
    std::cout << "Hallo Welt" << std::endl;
    return 0;
}
\end{lstlisting}

Para hacer el código mas corto debemos incluir \textit{using namespace std} con lo que le estamos diciendo al compilador que usaremos el espacio de nombres std por lo que no tendremos que incluirlo cuando usemos elementos de este espacio de nombres, como pueden ser los objetos cout y cin, que representan el flujo de salida estándar (típicamente la pantalla o una ventana de texto) y el flujo de entrada estándar (típicamente el teclado).

Se veria de la siguiente manera:

\begin{lstlisting}[style=Cpp, label=using-namespace-std, caption=Using Namespace Std]
#include <iostream>

using namespace std;

int main()
{
    cout << "Hola Mundo" << endl;
    cout << "Hello World" << endl;
    cout << "Hallo Welt" << endl;
    return 0;
}
\end{lstlisting}

Los pasos siguientes son para una compilación en GNU o sistema operativo Unix, para generar el ejecutable del programa se compila con g++ de la siguiente forma:

\begin{listing}[style=consola, numbers=none,label=compilar,caption=Compilar]
| g++ hola.cpp -o hola
\end{listing}

Para poder ver los resultados del programa en acción, se ejecuta el programa de la siguiente forma:

\begin{listing}[style=consola, numbers=none,label=ejecutar,caption=Ejecutar]
|./hola
\end{listing}

Y a continuación se debe mostrar algo como lo siguiente:

\begin{listing}[style=consola, numbers=none,label=resultado,caption=Resultado]
Hola Mundo
\end{listing}

\subsection{Sintaxis}\index{Lo mas basico!Sintaxis}

Sintaxis es la forma correcta en que se deben escribir las instrucciones para el computador en un lenguaje de programación específico. C++ hereda la sintaxis de C estándar, es decir, la mayoría de programas escritos para el C estándar pueden ser compilados en C++.

\textbf{El punto y coma}

El punto y coma es uno de los simbólos más usados en C, C++; y se usa con el fin de indicar el final de una línea de instrucción. El punto y coma es de uso obligatorio.

\textbf{Ejemplo}

\begin{lstlisting}[style=Cpp, label=sintaxis, caption=Sintaxis]
clrscr(); //Limpiar pantalla, funciona con la librería conio de Borland C++
x = a + b;
 
string IP = "127.0.0.1"; // Variable IP tipo string
cout << IP << endl; // Devuelve 127.0.0.1
 
char Saludo[5] = "Hola"; // Variable Saludo tipo char
cout << Saludo[0] << endl; // Igual a H
cout << Saludo[1] << endl; // Igual a o
cout << Saludo[2] << endl; // Igual a l
cout << Saludo[3] << endl; // Igual a a
\end{lstlisting}

El punto y coma se usa también para separar contadores, condicionales e incrementadores dentro de un sentencia for

\textbf{Ejemplo}

\begin{lstlisting}[style=Cpp, label=sintaxis, caption=Sintaxis]
for (i=0; i < 10; i++) cout << i;
\end{lstlisting}

\textbf{Espacios y tabuladores}

Usar caracteres extras de espaciado o tabuladores ( caracteres tab ) es un mecanismo que nos permite ordenar de manera más clara el código del programa que estemos escribiendo, sin embargo, el uso de estos es opcional ya que el compilador ignora la presencia de los mismos. Por ejemplo, el segundo de los ejemplos anteriores se podría escribir como:

\begin{lstlisting}[style=Cpp, label=sintaxis, caption=Sintaxis]
for (int i=0; i < 10; i++) { cout << i * x; x++; }
\end{lstlisting}

y el compilador no pondría ningún reparo.


\subsection{Comentarios}\index{Lo mas basico!Comentarios}

Existen dos modos basicos para comentar en c++:
\begin{description}
\item [//] \hfill \\
Comentan solo una linea de código 
\item [/*Comentario*/] \hfill \\
Comentan estrofas de código
\end{description}

A continuación un ejemplo:

\begin{lstlisting}[style=Cpp, label=comentarios, caption=Comentarios]
#include <iostream>

using namespace std;

int main()
{
    /*
    cout    es para imprimir
    <<      se utiliza para separar elementos para cout
    endl    es un salto de linea
    */

    //En español
    cout << "Hola Mundo" << endl;
    //En ingles
    cout << "Hello World" << endl;
    //En aleman
    cout << "Hallo Welt" << endl;
    return 0;
}
\end{lstlisting}

\subsection{Datos primitivos}\index{Lo mas basico!Datos primitivos}

En un lenguaje de programación es indispensable poder almacenar información, para esto en C++ están disponibles los siguientes tipos que permiten almacenar información numérica de tipo entero o real:

\begin{center}
	\begin{tabular}{|c|c|c|c|}
		\hline
		Nombre & Descripción & Tamaño & Rango de valores\\
		\hline
		bool & Valor booleano & 1byte & true o false\\
		\hline
		char & Carácter o entero pequeño & 1byte & De -128 a 127\\
		\hline
		short int & Entero corto & 2bytes & De -32768 a 32767\\
		\hline
		int & Entero & 4bytes & De -2147483648 a 2147483647\\
		\hline
		long long & Entero largo & 8bytes & -9223372036854775808 a 9223372036854775807\\
		\hline
		float & Número de punto flotante & 4bytes & 3.4e +/- 38 (7 digitos)\\
		\hline
		double & Float con doble precisión & 8bytes & 1.7e +/- 308 (15 digitos)\\
		\hline
	\end{tabular}
\end{center}

Los valores dependen de la arquitectura utilizada. Los mostrados son los que generalmente se encuentran en una máquina típica de arquitectura 32 bits.

\textbf{El Modificador unsigned}

El modificador unsigned es utilizado únicamente con los enteros, su utilización permite utilizar en los enteros únicamente la parte positiva,

\begin{lstlisting}[style=Cpp, label=unsigned, caption=Unsigned]
int a; 			// Almacena valores entre -32768 y 32767
unsigned int a; // Almacena valores entre 0 y 65535
\end{lstlisting}


\subsection{Variables y constantes}\index{Lo mas basico!Variables y constantes}

Una variable, como su nombre lo indica, es un determinado objeto cuyo valor puede cambiar durante el proceso de una tarea específica. Contrario a una variable, una constante es un determinado objeto cuyo valor no puede ser alterado durante el proceso de una tarea específica. En C, C++ para declarar variables no existe una palabra especial, es decir, las variables se declarán escribiendo el tipo seguido de uno o más identificadores o nombres de variables. Por otro lado, para declarar constantes existe la palabra reservada const, así como la directiva \textit{\#define}. A continuación se muestran ejemplos de declaración de variables y constantes.
\begin{center}
	\begin{tabular}{|c|c|c|}
		\hline
		Variables & Constantes & Constantes\\
		\hline
		int a; & const int a = 100;	&  \#define a 100\\
		\hline
		float b; & const float b = 100; & \#define b 100\\
		\hline
	\end{tabular}
\end{center}

\textbf{Notas:}

A diferencia de las constantes declaradas con la palabra const los símbolos definidos con \textit{\#define} no ocupan espacio en la memoria del código ejecutable resultante.

El tipo de la variable o constante puede ser cualquiera de los listados en Tipos primitivos, o bien de un tipo definido por el usuario.

Las constantes son usadas a menudo con un doble propósito, el primero es con el fin de hacer más legible el código del programa, es decir, si se tiene (por ejemplo) la constante numerica 3.1416 y esta representa al número pi, entonces podemos hacer declaraciones tales como:

\begin{lstlisting}[style=Cpp, label=constantes, caption=Constantes]
#define pi 3.1416
\end{lstlisting}

En este caso podremos usar la palabra pi en cualquier parte del
programa y el compilador se encargará de cambiar dicho simbolo por 3.1416.
o bien,

\begin{lstlisting}[style=Cpp, label=constantes, caption=Constantes]
const pi = 3.1416;
\end{lstlisting}

En este otro caso podremos usar la palabra pi en cualquier parte del programa y el compilador se encargará
de cambiar dicho símbolo por una referencia a la constante pi guardada en la memoria.

\subsection{Lectura e Impresión}\index{Lo mas basico!Lectura e Impresión}

\textbf{La iostream}

La iostream es la biblioteca estándar en C++ para poder tener acceso a los dispositivos estándar de entrada y/o salida. En sus programas, si usted desea hacer uso de los objetos cin, cout, cerr y clog tendrá que incluir ( por medio de la directiva \textit{\#include} ) el uso de la biblioteca iostream. En la iostream se encuentran definidas las clases ios ( misma que es la base para las clases que implementen operaciones de entrada y/o salida de datos ), istream ( para operaciones de entrada ) y ostream ( para operaciones de salida ). Aparte de las clases mencionadas, en la iostream se encuentra una lista de variables y constantes ( atributos ) que son accesibles por el usuario a través del operador de ámbito ( :: ) si es que no se incluyera \textit{using namespace std}.

Si usted usa la directiva \#include <iostream.h> o \#include <iostream> en sus programas, automáticamente la iostream pone a su disposición los objetos cin, cout, clog y cerr en el ámbito estándar (std), de tal manera que usted puede comenzar a enviar o recibir información a través de los mismos sin siquiera preocuparse de su creación. Asi, un sencillo ejemplo del uso de los objetos mencionados se muestra en seguida.

\begin{lstlisting}[style=Cpp, label=iostream, caption=Iostream]
// De nuevo con el hola mundo...
#include <iostream.h>
int main()
{
    std::cout << "Hola mundo";    // imprimir mensaje (en la pantalla)
    std::cin.get();    // lectura ( entrada del teclado )
    return 0;
}
\end{lstlisting}

\textbf{Operadores de direccionamiento}

Los operadores de direccionamiento son los encargados de manipular el flujo de datos desde o hacia el dispositivo referenciado por un stream específico. El operador de direccionamiento para salidas es una pareja de símbolos de "menor que" <<, y el operador de direccionamiento para entradas es una pareja de símbolos de "mayor que" >>. Los operadores de direccionamiento se colocan entre dos operandos, el primero es el Stream y el segundo es una variable o constante que proporciona o recibe los datos de la operación. Por ejemplo, en el siguiente programa y en la instrucción cout << "Entre su nombre: "; la constante "Entre su nombre: " es la fuente o quien proporciona los datos para el objeto cout. Mientras que en la instrucción cin >> nombre la variable nombre es el destino o quien recibe los datos provenientes del objeto cin.

\begin{lstlisting}[style=Cpp, label=operdirec, caption=Oper Direc]
// De nuevo con el hola mundo...
#include <iostream.h>
int main()
{
    char nombre[80];
    cout << "Entre su nombre: ";
    cin  >> nombre;
    cout << "Hola," << nombre;
    cin.get();
    return 0;
}
\end{lstlisting}

Observe que si en una misma línea de comando se desea leer o escribir sobre varios campos a la vez, no es necesario nombrar más de una vez al stream. Ejemplos:

\begin{lstlisting}[style=Cpp, label=ejemplo-ioc, caption=Ejemplo Cin Cout]
cout << "Hola," << nombre;
cin >> A >> B >> C;
\end{lstlisting}

Otro ejemplo del manejo de \textit{cin}:

\begin{lstlisting}[style=Cpp, label=impresión, caption=Impresión]
// Programa que muestra diversos textos por consola

// Las librerías del sistema usadas son las siguientes
#include <iostream>

using namespace std;
// Es la función principal encargada de mostrar por consola diferentes textos
int main(void)
{
    // Ejemplo con una única línea, se muestra el uso de cout y endl
    cout << "Bienvenido. Soy un programa. Estoy en una linea de codigo." << endl;

    // Ejemplo con una única línea de código que se puede fraccionar
    // mediante el uso de '<<'
    cout << "Ahora "
         << "estoy fraccionado en el codigo, pero en la consola me muestro como una unica frase."
         << endl;

    // Uso de un código largo, que cuesta leer para un programador,
    // y que se ejecutará sin problemas.
    // *** No se recomienda hacer líneas de esta manera,
    // esta forma de programar no es apropiada ***
    cout << "Un gran texto puede ocupar muchas lineas."
         << endl
         << "Pero eso no frena al programador a que todo se pueda poner en una unica linea de codigo y que"
         << endl
         << "el programa, al ejecutarse, lo situe como el programador quiso"
         << endl;

    return 0; // Y se termina con éxito.
}
\end{lstlisting}

Consola:

\begin{listing}[style=consola, numbers=none,label=resultado-impresión,caption=Resultado Impresión]
Bienvenido. Soy un programa. Estoy en una linea de codigo.
Ahora estoy fraccionado en el codigo, pero en la consola me muestro como una unica frase.
Un gran texto puede ocupar muchas lineas.
Pero eso no frena al programador a que todo se pueda poner en una unica linea de codigo y que
el programa, al ejecutarse, lo situe como el programador quiso

\end{listing}


\subsection{Operadores}\index{Lo mas basico!Operadores}

El C++ está lleno de operadores. Presentamos aquí una tabla de los mismos indicando el rango de prioridad de cada uno, y cómo se ejecutan. A continuación comentaremos brevemente los operadores.

\begin{center}
	\begin{tabular}{|c|c|}
		\hline
		\textbf{Operadores (de mayor a menor prioridad)} & \textbf{Sentido} \\
		\hline
		( ) \{ \} & I-D \\
		\hline
		! ++ --  * \& & D-I \\
		\hline
		* / \% & I-D \\
		\hline
		+ - & I-D \\
		\hline
		<< >> & I-D \\
		\hline
		< <= > >= & I-D \\
		\hline
		== != & I-D \\
		\hline
		\& & I-D \\
		\hline
		\textasciicircum & I-D \\
		\hline
		| & I-D \\
		\hline
		\&\& (and) & I-D \\
		\hline
		|| (or) & I-D \\
		\hline
		?: (Operador Condicional Ternario) & I-D \\
		\hline
		= += -= *= /= \%= & D-I \\
		\hline
	\end{tabular}
\end{center}

La acción de estos operadores es la siguiente:

\subsubsection{Operadores Aritméticos}\index{Lo mas basico!Operadores!Operadores Aritméticos}

\begin{description}
\item [+] Suma los valores situados a su derecha y a su izquierda.
\item [-] Resta el valor de su derecha del valor de su izquierda.
\item [-] Como operador unario, cambia el signo del valor  de su izquierda.
\item [*] Multiplica el valor de su derecha por el valor de su izquierda.
\item [/] Divide el valor situado a su izquierda por el valor situado a su dercha.
\item [\%] Proporciona el resto de la división del valor de la izquierda por el valor de la derecha (sólo enteros).
\item [++] Suma $1$ al valor de la variable situada a su izquierda (modo prefijo) o de la variable situada a su derecha (modo sufijo).
\item [--] Igual que ++, pero restando $1$. 
\end{description}

\subsubsection{Operadores de Asignación}\index{Lo mas basico!Operadores!Operadores de Asignación}

\begin{description}
\item [=] Asigna el valor de su derecha a la variable de su izquierda.
\end{description}

Cada uno de los siguientes operadores actualiza la variable de su izquierda con el valor de su derecha utilizando la operación indicada. Usaremos de \textit{d} e \textit{i} para iquierda.

\begin{description}
\item [+=] Suma la cantidad \textit{d} a la variable \textit{i}.
\item [-=] Resta la cantidad \textit{d} de la variable \textit{i}.
\item [*=] Multiplica la variable \textit{i} por la variable \textit{d}.
\item [/=] Divide la variable \textit{i} entre la cantidad \textit{d}.
\item [\%=] Proporciona el resto de la división de la variable \textit{i}  por la cantidad \textit{d}
\end{description}

\textbf{Ejemplo}

conejos *= 1.6; //es lo mismo que conejos = conejos * 1.6;

\subsubsection{Operadores de Relación}\index{Lo mas basico!Operadores!Operadores de Relación}

Cada uno de estos operadores compara el valor de su izquierda con el valor de su dercha. La expresión de relación formada por un operador y sus dos operandos toma el valor $1$ si la expresión es cierta, y el valor $0$ si es falsa.

\begin{description}
\item [<] menor que
\item [<=] menor o igual que 
\item [==] igual a
\item [>=] mayor o igual que
\item [>] mayor que 
\item [!=] distinto de
\end{description}

\subsubsection{Operadores Lógicos}\index{Lo mas basico!Operadores!Operadores Lógicos}

Los operadores lógicos utilizan normalmenteexpresiones de relación como operadores. El operador \textit{!} toma un operando situado a su derecha; el resto toma dos: uno a su derecha y otro a su izquierda.

\begin{description}
\item [\&\& and] La expresión combinada es cierta si ambos operandos lo son, y falsa en cualquier otro caso.
\item [|| or] La expresión combinada es cierta si uno o ambos operandos lo son, y falsa en cualquier otro caso.
\item [! not] La expresión es cierta si el operador es falso, y viceversa.
\end{description}

\subsubsection{Operadores Relacionados con punteros}\index{Lo mas basico!Operadores!Operadores Relacionados con punteros}

\begin{description}
\item [\& Operador de Dirección] Cuando va seguido por el nombre de una variable, entrega la dirección de dicha variable \textit{\&abc} es la dirección de la variable \textit{abc}.
\item [* Operador de Indirección] Cuando va seguido por un puntero, entrega el valor almacenado en la dirección apuntada por él \textit{abc = 22; def = \&abc; val = *def} el efecto es asignar a \textit{val} el valor de 22.
\end{description}

\subsubsection{Operadores de Estructuras y Uniones}\index{Lo mas basico!Operadores!Operadores de Estructuras y Uniones}

\subsubsection{Operadores Lógicos de Bits}\index{Lo mas basico!Operadores!Operadores Lógicos de Bits}
\subsubsection{Misceláneas}\index{Lo mas basico!Operadores!Misceláneas}







\section{Estructuras de Control}\index{Estructuras de Control}

Las estructuras de control permiten modificar el flujo de ejecución de las instrucciones de un programa.

Con las estructuras de control se puede:

\begin{itemize}
\item De acuerdo a una condición, ejecutar un grupo u otro de sentencias (\textbf{If})
\item De acuerdo al valor de una variable, ejecutar un grupo u otro de sentencias (\textbf{Switch})
\item Ejecutar un grupo de sentencias mientras se cumpla una condición (\textbf{While})
\item Ejecutar un grupo de sentencias hasta que se cumpla una condición (\textbf{Do-While})
\item Ejecutar un grupo de sentencias un número determinado de veces (\textbf{For})
\end{itemize}

Todas las estructuras de control tienen un único punto de entrada y un único punto de salida.


\subsection{Sentencias de decisión}\index{Estructuras de Control!Sentencias de decisión}

\begin{center}
	\textbf{Definición}
\end{center}

Las sentencias de decisión o también llamadas de CONTROL DE FLUJO son estructuras de control que realizan una pregunta la cual retorna verdadero o falso (evalúa una condicion) y selecciona la siguiente instrucción a ejecutar dependiendo la respuesta o resultado.


\subsubsection{Sentencia if}\index{Estructuras de Control!Sentencias de decisión!Sentencia if}

La instrucción if es, por excelencia, la más utilizada para construir estructuras de control de flujo.

\begin{center}
	\textbf{Sintaxis}
\end{center}

\textbf{Primera Forma}

Ahora bién, la sintaxis utilizada en la programación de C++ es la siguiente:

\begin{lstlisting}[style=Cpp, label=sintaxis-if, caption=Sintaxis If]
if (condicion)
   {
       Set de instrucciones
   }
\end{lstlisting}

siendo \textit{"condicion"} el lugar donde se pondrá la condición que se tiene que cumplir para que sea verdadera la sentencia y así proceder a realizar el \textit{"set de instrucciones"} o código contenido dentro de la sentencia.

\textbf{Segunda Forma}

Ahora veremos la misma sintaxis pero ahora le añadiremos la parte \textit{"Falsa"} de la sentencia:

\begin{lstlisting}[style=Cpp, label=sintaxis-if-else, caption=Sintaxis If Else]
if (condicion)
   {
       Set de instrucciones    //Parte VERDADERA
   }
else
   {
       Set de instrucciones 2  //Parte FALSA
   }
\end{lstlisting}

La forma mostrada anteriormente muestra la union de la parte \textit{"VERDADERA"} con la nueva secuencia la cual es la parte \textit{"FALSA"} de la sentencia de decision \textit{"IF"} en la cual esta compuesta por el:

\begin{lstlisting}[style=Cpp, label=sintaxis-else, caption=Sintaxis Else]
else
   {
     Set de instrucciones 2  //Parte FALSA
   }
\end{lstlisting}

la palabra \textit{"else"} o \textit{"De lo contrario"} indica al lenguaje que de lo contrario al no ser verdadera o no se cumpla la parte verdadera entonces realizara el \textit{"set de instrucciones 2"}.

\begin{center}
	\textbf{Ejemplos De Sentencias If}
\end{center}

\textbf{Ejemplo 1:}

\begin{lstlisting}[style=Cpp, label=if-ejemplo-1, caption=If Ejemplo 1]
if(numero == 0) //La condicion indica que tiene que ser igual a Cero
    {
      cout<<"El Numero Ingresado es Igual a Cero";
    }
\end{lstlisting}

\textbf{Ejemplo 2:}

\begin{lstlisting}[style=Cpp, label=if-ejemplo-2, caption=If Ejemplo 2]
if(numero > 0) // la condicion indica que tiene que ser mayor a Cero
    {    
      cout<<"El Numero Ingresado es Mayor a Cero";
    }
\end{lstlisting}

\textbf{Ejemplo 3:}

\begin{lstlisting}[style=Cpp, label=if-ejemplo-3, caption=If Ejemplo 3]
if(numero < 0) // la condicion indica que tiene que ser menor a Cero
    {    
      cout<<"El Numero Ingresado es Menor a Cero";
    }
\end{lstlisting}

Ahora uniremos todos estos ejemplos para formar un solo programa mediante la utilización de la sentencia "Else" e introduciremos el hecho de que se puede escribir en este espacio una sentencia if ya que podemos ingresar cualquier tipo de código dentro de la sentencia escrita después de un Else.

\textbf{Ejemplo 4:}

\begin{lstlisting}[style=Cpp, label=if-ejemplo-4, caption=If Ejemplo 4]
if(numero == 0) //La condicion indica que tiene que ser igual a Cero
   {
     cout<<"El Numero Ingresado es Igual a Cero";
   }
else
   {
      if(numero > 0) // la condicion indica que tiene que ser mayor a Cero
      {    
          cout<<"El Numero Ingresado es Mayor a Cero";
      }
      else
      {
         if(numero < 0) // la condicion indica que tiene que ser menor a Cero
         {    
           cout<<"El Numero Ingresado es Menor a Cero";
         }
      }
    }
\end{lstlisting}


\subsubsection{Sentencia switch}\index{Estructuras de Control!Sentencias de decisión!Sentencia switch}

switch es otra de las instrucciones que permiten la construcción de estructuras de control. A diferencia de if, para controlar el flujo por medio de una sentencia switch se debe de combinar con el uso de las sentencias \textit{case} y \textit{break}.

\textbf{Notas:} cualquier número de casos a evaluar por switch así como la sentencia default son opcionales. La sentencia switch es muy útil en los casos de presentación de menus.

\begin{center}
	\textbf{Sintaxis}
\end{center}
Ahora bién, la sintaxis utilizada en la programación de C++ es la siguiente:

\begin{lstlisting}[style=Cpp, label=sintaxis-switch, caption=Sintaxis Switch]
switch (condicion)
{
    case primer_caso:
         bloque de instrucciones 1
    break;

    case segundo_caso:
         bloque de instrucciones 2
    break;

    case caso_n:
         bloque de instrucciones n
    break;

    default: bloque de instrucciones por defecto
}
\end{lstlisting}

\begin{center}
	\textbf{Ejemplos De Sentencias Switch}
\end{center}

\textbf{Ejemplo 1}

\begin{lstlisting}[style=Cpp, label=switch-ejemplo-1, caption=Switch Ejemplo 1]
switch (numero)
{
    case 0: cout << "numero es cero";
}
\end{lstlisting}

\textbf{Ejemplo 2}

\begin{lstlisting}[style=Cpp, label=switch-ejemplo-2, caption=Switch Ejemplo 2]
switch (opcion)
{
    case 0: cout << "Su opcion es cero"; break;
    case 1: cout << "Su opcion es uno"; break;
    case 2: cout << "Su opcion es dos";
}
\end{lstlisting}

\textbf{Ejemplo 3}

\begin{lstlisting}[style=Cpp, label=switch-ejemplo-3, caption=Switch Ejemplo 3]
switch (opcion)
{
    case 1: cout << "Su opcion es 1"; break;
    case 2: cout << "Su opcion es 2"; break;
    case 3: cout << "Su opcion es 3"; break;
    default: cout << "Elija una opcion entre 1 y 3";
}
\end{lstlisting}


\subsubsection{Operador condicional ternario ?:}\index{Estructuras de Control!Sentencias de decisión!Operador condicional ternario ?:}

En C/C++, existe el operador condicional ( ?: ) el cual es conocido por su estructura como ternario. El comportamiento de dicho operador es el mismo que una estructura if - then - else del lenguaje BASIC (y de la función IIf de Visual Basic). El operador condicional ?: es útil para evaluar situaciones tales como:

Si se cumple tal condición entonces haz esto, de lo contrario haz esto otro.

\begin{center}
	\textbf{Sintaxis}
\end{center}

\begin{lstlisting}[style=Cpp, label=sintaxis-switch, caption=Sintaxis Switch]
( (condicion) ? proceso1 : proceso2 )
\end{lstlisting}

En donde, condicion es la expresión que se evalua, proceso1 es la tarea a realizar en el caso de que la evaluación resulte verdadera, y proceso2 es la tarea a realizar en el caso de que la evaluación resulte falsa.

\begin{center}
	\textbf{Ejemplos De Sentencias Operador condicional ternario ?:}
\end{center}

\textbf{Ejemplo 1}

\begin{lstlisting}[style=Cpp, label=condicional-ejemplo-1, caption=Condicional Ejemplo 1]
int edad;
cout << "Cual es tu edad: ";
cin >> edad;
cout << ( (edad < 18) ? "Eres joven aun" : "Ya tienes la mayoria de edad" );
El ejemplo anterior podria escribirse de la siguiente manera:

int edad;
cout << "Cual es tu edad: ";
cin >> edad;
if (edad < 18) cout << "Eres joven aun";
    else cout << "Ya tienes la mayoria de edad";
\end{lstlisting}

\textbf{Ejemplo 2}

Vamos a suponer que deseamos escribir una función que opere sobre dos valores numéricos y que la misma ha de regresar 1 (true) en caso de que el primer valor pasado sea igual al segundo valor; en caso contrario la función debe retornar 0 (false).

\begin{lstlisting}[style=Cpp, label=condicional-ejemplo-2, caption=Condicional Ejemplo 2]
int es_igual( int a, int b)
{
    return ( (a == b) ? 1 : 0 )
}
\end{lstlisting}

\subsection{Sentencias de iteración}\index{Estructuras de Control!Sentencias de iteración}

\begin{center}
	\textbf{Definición}
\end{center}

Las Sentencias de Iteración o Ciclos son estructuras de control que repiten la ejecución de un grupo de instrucciones. Básicamente, una sentencia de iteración es una estructura de control condicional, ya que dentro de la misma se repite la ejecución de una o más instrucciones mientras que una a condición especifica se cumpla. Muchas veces tenemos que repetir un número definido o indefinido de veces un grupo de instrucciones por lo que en estos casos utilizamos este tipo de sentencias. en C++ los ciclos o bucles se construyen por medio de las sentencias for, while y do - while. La sentencia for es útil para los casos en donde se conoce de antemano el número de veces que una o más sentencias han de repetirse. Por otro lado, la sentencia while es útil en aquellos casos en donde no se conoce de antemano el número de veces que una o más sentencias se tienen que repetir.

\subsubsection{Sentencias For}\index{Estructuras de Control!Sentencias de iteración!Sentencias For}

Ahora bién, la sintaxis utilizada en la programación de C++ es la siguiente:

\begin{center}
	\textbf{Sintaxis}
\end{center}

\begin{lstlisting}[style=Cpp, label=sintaxis-for, caption=Sintaxis For]
for(contador; final; incremento)
{
    Codigo a Repetir;
}
\end{lstlisting}

donde:

\begin{description}
\item [contador] es una variable numérica
\item [final] es la condición que se evalua para finalizar el ciclo (puede ser independiente del contador)
\item [incremento] es el valor que se suma o resta al contador
\end{description}

Hay que tener en cuenta que el \textit{"for"} evalua la condición de finalización igual que el while, es decir, mientras esta se cumpla continuaran las repeticiones.

\begin{center}
	\textbf{Ejemplos De Sentencias For}
\end{center}

\textbf{Ejemplo 1}

\begin{lstlisting}[style=Cpp, label=for-ejemplo-1, caption=For Ejemplo 1]
for(int i=1; i<=10; i++)
{
    cout<<"Hola Mundo";   
}
\end{lstlisting}

Esto indica que el contador \textit{"i"} inicia desde 1 y continuará iterando mientras i sea menor o igual a 10 ( en este caso llegará hasta 10) e \textit{"i++"} realiza la sumatoria por unidad lo que hace que el for y el contador se sumen. repitiendo 10 veces \textit{"HOLA MUNDO"} en pantalla.

\textbf{Ejemplo 2}

\begin{lstlisting}[style=Cpp, label=for-ejemplo-2, caption=For Ejemplo 2]
for(int i=10; i>=0; i--)
{
    cout<<"Hola Mundo";   
}
\end{lstlisting}

Este ejemplo hace lo mismo que el primero, salvo que el contador se inicializa a 10 en lugar de 1; y por ello cambia la condición que se evalua así como que el contador se decrementa en lugar de ser incrementado.

La condición también puede ser independiente del contador:

\textbf{Ejemplo 3}

\begin{lstlisting}[style=Cpp, label=for-ejemplo-3, caption=For Ejemplo 3]
int j = 20;
for(int i=0; j>0; i++){
    cout<<"Hola"<<i<<" - "<<j<<endl;
    j--;
}
\end{lstlisting}

En este ejemplo las iteraciones continuaran mientras j sea mayor que 0, sin tener en cuenta el valor que pueda tener i.


\subsubsection{Sentencia While}\index{Estructuras de Control!Sentencias de iteración!Sentencia While}

Ahora bién, la sintaxis utilizada en la programación de C++ es la siguiente:

\begin{center}
	\textbf{Sintaxis}
\end{center}

\begin{lstlisting}[style=Cpp, label=sintaxis-while, caption=Sintaxis While]
while(condicion)
{
    codigo a Repetir
}
\end{lstlisting}

donde:

\begin{description}
\item [condicion] es la expresión a evaluar
\end{description}

\begin{center}
	\textbf{Ejemplos De Sentencias While}
\end{center}

\textbf{Ejemplo 1}

\begin{lstlisting}[style=Cpp, label=while-ejemplo-1, caption=While Ejemplo 1]
int contador = 0;
 
while(contador<=10)
{
    contador=contador+1;
    cout<<"Hola Mundo";
}
\end{lstlisting}

El contador Indica que hasta que este llegue a el total de 10 entonces se detendrá y ya no se realizará el código contenido dentro de la sentencia while, de lo contrario mientras el \textit{"contador"} sea menor a 10 entonces el código contenido se ejecutará desplegando hasta 10 veces "Hola Mundo" en pantalla.


\subsubsection{Sentencia Do - While}\index{Estructuras de Control!Sentencias de iteración!Sentencia Do - While}

\begin{center}
	\textbf{Sintaxis}
\end{center}

La sentencia do es usada generalmente en cooperación con while para garantizar que una o más instrucciones se ejecuten al menos una vez. Por ejemplo, en la siguiente construcción no se ejecuta nada dentro del ciclo while, el hecho es que el contador inicialmente vale cero y la condición para que se ejecute lo que está dentro del while es \textit{"mientras el contador sea mayor que diez"}. Es evidente que a la primera evaluación hecha por while la condición deja de cumplirse.

\begin{lstlisting}[style=Cpp, label=sintaxis-do-while, caption=Sintaxis Do While]
int contador = 0;
 
while(contador > 10)
{
    contador ++;
    cout<<"Hola Mundo";
}
\end{lstlisting}

Al modificar el segmento de código anterior usando do tenemos:

\begin{lstlisting}[style=Cpp, label=sintaxis-do-while, caption=Sintaxis Do While]
int contador = 0;
 
do
{
    contador ++;
    cout<<"Hola Mundo";
}
while(contador > 10);
\end{lstlisting}

Observe cómo en el caso de do la condición es evaluada al final en lugar de al principio del bloque de instrucciones y, por lo tanto, el código que le sigue al \textit{do} se ejecuta al menos la primera vez.


\subsection{Sentencias Break y Continue}\index{Estructuras de Control!Sentencias Break y Continue}

En la sección (Sentencia switch) vimos que la sentencia break es utilizada con el propósito de forzar un salto dentro del bloque switch hacia el final del mismo. En esta sección volveremos a ver el uso de break, salvo que esta ocasión la usaremos junto con las sentecias for y la sentencia while. Además, veremos el uso de la sentencia continue.


\subsubsection{Break}\index{Estructuras de Control!Sentencias de iteración!Sentencias Break y Continue!Break}

La sentencia break se usa para forzar un salto hacia el final de un ciclo controlado por for o por while.

\begin{center}
	\textbf{Ejemplo}
\end{center}

En el siguiente fragmento de código la sentencia break cierra el ciclo for cuando la variable ( i ) es igual a 5. 

\begin{lstlisting}[style=Cpp, label=break, caption=Break]
for (int i=0; i<10; i++) {
    if (i == 5) break;
    cout << i << " ";
}
\end{lstlisting}

La salida para el mismo será:

\begin{listing}[style=consola, numbers=none,label=resultado-break,caption=Resultado Break]
0 1 2 3 4
\end{listing}


\subsubsection{Continue}\index{Estructuras de Control!Sentencias de iteración!Sentencias Break y Continue!Continue}

La sentencia continue se usa para ignorar una iteración dentro de un ciclo controlado por for o por while.

\begin{center}
	\textbf{Ejemplo}
\end{center}

En el siguiente fragmento de código la sentencia continue ignora la iteración cuando la variable ( i ) es igual a 5.

\begin{lstlisting}[style=Cpp, label=continue, caption=Continue]
for (int i=0; i<10; i++) {
    if (i == 5) continue;
    cout << i << " ";
}
\end{lstlisting}

La salida para el mismo será:

\begin{listing}[style=consola, numbers=none,label=resultado-continue,caption=Resultado Continue]
0 1 2 3 4 6 7 8 9
\end{listing}

\subsubsection{Uso de break y continue junto con while}\index{Estructuras de Control!Sentencias de iteración!Sentencias Break y Continue!Uso de break y continue junto con while}

Los dos ejemplos anteriores se presentan en seguida, salvo que en lugar de for se hace uso de while.

\textbf{Nota:} no deje de observar que la construcción del ciclo while para el caso de la sentencia continue es diferente, esto para garantizar que el ciclo no vaya a caer en una iteración infinita.

\begin{center}
	\textbf{Break}
\end{center}

\begin{lstlisting}[style=Cpp, label=break-while, caption=Break While]
int i = 0;
while (i<10) {
    if (i == 5) break;
    cout << i << " ";
    i++;
}
\end{lstlisting}

\begin{center}
	\textbf{Continue}
\end{center}

\begin{lstlisting}[style=Cpp, label=continue-while, caption=Continue While]
int i = -1;
while (i<10) {
    i++;
    if (i == 5) continue;
    cout << i << " ";
}
\end{lstlisting}

%----------------------------------------------------------------------------------------
%	BIBLIOGRAPHY
%----------------------------------------------------------------------------------------

\chapter*{Bibliography}
\addcontentsline{toc}{chapter}{\textcolor{mycolor}{Bibliography}}
\section*{Books}
\addcontentsline{toc}{section}{Books}
\printbibliography[heading=bibempty,type=book]
\section*{Articles}
\addcontentsline{toc}{section}{Articles}
\printbibliography[heading=bibempty,type=article]

%----------------------------------------------------------------------------------------
%	INDEX
%----------------------------------------------------------------------------------------

\cleardoublepage
\phantomsection
\setlength{\columnsep}{0.75cm}
\addcontentsline{toc}{chapter}{\textcolor{mycolor}{Index}}
\printindex

%----------------------------------------------------------------------------------------

%Listar las estrofas de códigos
\lstlistoflistings

\end{document}